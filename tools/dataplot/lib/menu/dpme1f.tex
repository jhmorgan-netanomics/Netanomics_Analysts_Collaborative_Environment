MENU FILE #1 (DATAPLOT GENERAL)
 
14                 0       167
Hardware, CPU's, Operating Systems, etc.
0                  0       300
DATAPLOT
1                397         0
Overview
2                497         0
Numbers, Parameters, Variables, Matrices, Functions/Strings
2.1              497        11
Numbers (named scalars)
2.1.1            497        20
Storage of Numbers
2.1.2            497        27
Conventions for Writing Numbers
2.1.3            497        49
Operating on Numbers via the LET Command
2.1.4            497        59
The Use of Numbers in Analyses
2.2              497        71
Parameters (named scalars)
2.2.1            497        88
Naming Conventions for Parameters
2.2.2            497       103
Storage of Parameter Values
2.2.3            497       110
Creating Parameters via the LET Command
2.2.4            497       120
Creating Parameters via the READ PARAMETER Command
2.2.5            497       135
Writing Parameters to Screen & File via the WRITE Command
2.2.6            497       146
Neat Writing of Parameters via SET WRITE DECIMALS Command
2.2.7            497       169
Text/Title/Label/Legend Writing of Parameters via ^
2.2.7.1          497       185
Text Output via WRITE
2.2.7.2          497       192
General Text Output via TEXT
2.2.7.3          497       202
Title Output on Plots via TITLE
2.2.7.4          497       213
Horizontal or Vertical Label Output on Plots via ...LABEL
2.2.7.5          497       227
Legend Output on Plots via LEGEND
2.2.8            497       241
Neat Text/Title/Label/Legend Writing of Parameters via ROUND()
2.2.8.1          497       256
Text Output via WRITE
2.2.8.2          497       266
General Text Output via TEXT
2.2.8.3          497       278
Title Output on Plots via TITLE
2.2.8.4          497       290
Horizontal or Vertical Label Output on Plots via ...LABEL
2.2.8.5          497       307
Legend Output on Plots via LEGEND
2.2.9            497       324
Changing Values of Parameters via the LET Command
2.2.10           497       340
Deleting Parameters--Not Doable.
2.2.11           497       350
Automatically-Provided Parameters--PI and INFININTY
2.2.12           497       361
The Use of Parameters in DATAPLOT Analyses
2.3              497       383
Variables (named vectors = 1-dimensional arrays)
3               1032         0
Files, Data Files, Plot Files, etc.
4               1132         0
Mathematical and Relational Operators
5               1232         0
Commands
5.1             1232         9
Core Commands--PLOT, FIT & LET
5.2             1232        14
Grouped Alphabetically
5.3             1232        33
Grouped by Function
5.4             1232        45
Grouped By Discipline
6               1332         0
Subcommands Under LET
6.1             1332        10
Statistics & Probability
6.2             1332        33
Mathematics/General
6.3             1332        57
Mathematics/Transforms
6.4             1332        68
Mathematics/Vector, Matrix, Complex, Polynomial, etc.
6.5             1332        79
Other
7               1432         0
Reserved Words & Qualifiers
8               1532         0
Macros & Programs
9               1632         0
Built-in Functions
10              1638         0
Text Subcommands
11              1644         0
Auxiliary Files
11.1            1644        12
Reference Files
11.1.1          1644        27
Chemistry/Physics
11.1.2          1644        38
Geometry Formulae
11.1.3          1644        48
Probability Functions
11.1.4          1644        57
Ascii Numeric Equivalents
11.1.5          1644        65
Dataplot Default Settings
11.1.6          1644        73
Phone & Mail
11.1.7          1644        85
Business
11.1.8          1644       100
Quality Improvement/Taguchi/Japan
11.1.9          1644       119
Latitude of U.S. Cities
11.1.10         1644       127
Weather
11.2            1644       138
Data Files
Map Files       1644       144
   Dataplot has a few relatively coarse map data files.
11.4            1644       166
Design of Experiment Files
11.5            1644       172
Fractal Art Files
11.6            1644       195
Macro Files
11.7            1644       200
Menu Macro Files
11.7.1.         1644       207
What is a Menu Macro File and How Use?
11.7.2.         1644       228
List of Available Menu Macro Files
12              1992         0
Tutorial
----------      1992         6
 
14              2092         0
Hardware, CPU's, Operating Systems, etc.
15              2193         0
Graphics Drivers/Devices, Postscript, X-Windows, etc.
16              2294         0
Installation & Tailoring Hints
17              2395         0
Default Settings
18              2496         0
On-Line/Off-Line Help & Documentation
 
 
 
 
 
 
 
 
 
 
 
 
 
 
 
 
 
 
 
 
 
 
 
 
 
 
 
 
 
 
 
 
 
 
 
 
 
 
 
 
 
 
 
 
 
 
 
 
 
 
 
 
 
 
 
 
 
 
 
 
 
 
 
 
 
 
 
 
 
 
 
 
 
 
 
 
 
 
 
 
 
 
 
 
 
 
 
 
 
 
 
 
 
 
 
 
 
 
 
 
 
 
 
 
 
 
 
 
 
 
 
 
 
 
 
 
 
 
 
 
 
 
 
 
 
 
 
 
 
 
 
 
 
 
 
 
 
 
 
 
 
 
 
----------------------------------------
 
0
DATAPLOT
   1. Overview
   2. Numbers, Parameters, Variables, Matrices, Functions/Strings
   3. Files, Data Files, Plot Files, etc.
   4. Mathematical and Relational Operators
   5. Commands           (Go to top menu--enter T or TOP)
   6. Subcommands Under LET
   7. Reserved Words & Qualifiers
   8. Macros & Programs
   9. Built-in Functions (Go to top menu--enter T or TOP)
  10. Text Subcommands   (Go to top menu--enter T or TOP)
  11. Auxiliary Files
  12. Tutorial           (Go to top menu--enter T or TOP)
  13. Shortcuts/Saving Keystrokes
  14. Hardware, CPU's, Operating Systems, etc.
  15. Graphics Drivers/Devices, Postscript, X-Windows, etc.
  16. Installation & Tailoring Hints
  17. Default Settings
  18. On-Line/Off-Line Help & Documentation
 
----------------------------------------
 
 
 
 
 
 
 
 
 
 
 
 
 
 
 
 
 
 
 
 
 
 
 
 
 
 
 
 
 
 
 
 
 
 
 
 
 
 
 
 
 
 
 
 
 
 
 
 
 
 
 
 
 
 
 
 
 
 
 
 
 
 
 
 
 
 
 
 
 
 
 
 
 
----------  *OVERVIEW*  -------------------------
 
1
Overview
DATAPLOT is a powerful, flexible, interactive language/
system designed for use in a scientific/engineering
environement.  DATAPLOT is an interactive, command-driven
system with extensive core capabilities in
 
      1) graphics
 
and broad related capabilities in
 
      2) fitting /modeling
      3) general data analysis;
      4) mathematics.
 
DATAPLOT's areas of focus include raw graphics (2D,
3D, contour), analysis graphics, presentation graphics,
summary graphics, diagrammatic graphics, business graphics,
word charts, graphical data analysis, exploratory data
analysis, time series analysis, smoothing, fitting, data
analysis, statistics, probability, multivariate analysis,
design of experiment, statistical process control,
reliability, and mathematics.
 
DATAPLOT commands are high-level, English-syntax,
easy-to-learn, and self-descriptive, such as
 
      PLOT Y X
      PLOT EXP(-X**2) FOR X = -3 .1 3
      FIT Y = A+B*EXP(-ALPHA*X)
      BOX PLOT Y X
      ANOVA Y X1 X2 X3
      LET A = ROOTS SIN(X**2)+EXP(-X) FOR X = 0 TO 5
 
The 3 most important commands are PLOT, FIT, and LET.
The "average" analyst commonly uses about 20 commands.
The language as a whole consists of over 400 commands.
These 400+ commands are in 8 command categories--
 
   1) Graphics
   2) Diagrammatic Graphics
   3) Quantitative Analysis
   4) Plot Control
   5) Output Devices
   6) Input/Output
   7) Support
   8) Reserved Words
 
For syntax, default, etc.  information about an individual
command the analyst may--at any time--enter HELP
followed by the command name, as in
 
      HELP PLOT
      HELP FIT
      HELP LET
 
DATAPLOT's underlying code is 100% Fortran 77; it runs
on virtually all mainframes, minis, workstations, and
larger micros (386 PC's) including VAX/VMS, UNIX, DOS,
etc.  DATAPLOT has its own stand-alone graphics
subsystem with built-in graphics drivers for a wide
assortment of graphics devices including Tektronix,
HP-GL, POSTSCRIPT, X-Windows, etc.
 
DATAPLOT was written and developed at the National Institute
of Standards and Technology.  It was first implemented in
1977 and has had the benefit of continuous improvement &
enhancements in this superb scientific/engineering/research
environment.  The authors wish to thank the many NIST
scientists and engineers for their excellent suggestions
over the years in regard to DATAPLOT enhancements.
 
 
 
 
 
 
 
 
 
 
 
 
 
 
 
 
 
 
 
 
 
 
 
 
 
 
 
----------  *NUMBERS, PARAMETERS, ETC.*  ---------------------
 
2
Numbers, Parameters, Variables, Matrices, Functions/Strings
   DATAPLOT operates on the following 5 elements--
      1. Numbers (unnamed scalars)
      2. Parameters (named scalars)
      3. Variables (named vectors = 1-dimensional arrays)
      4. Matrices (named 2-dimensional arrays)
      5. Functions (named character strings)
 
----------------------------------------
 
2.1
Numbers (named scalars)
   1. Storage of Numbers
   2. Conventions for Writing Numbers
   3. Operating on Numbers via the LET Command
   4. The Use of Numbers in Analyses
 
----------------------------------------
 
2.1.1
Storage of Numbers
   1. All numbers are stored internally in single precision.
   2. Key intermediate calculations are in double precision.
 
----------------------------------------
 
2.1.2
Conventions for Writing Numbers
   1. A trailing decimal is optional and has no effect.
         LET A = 3.0
         LET A = 3.
         LET A = 3
         PLOT X FOR X = 1 1 10
   2. Trailing decimal zeros are optional and have no effect.
         LET B = 3.6500
         LET B = 3.650
         LET B = 3.65
         DEMODULATION FREQUENCY .4
   3. Leading zeros are optional and have no effect;
         LET C = 09.24
         LET C = 9.24
         FILTER WIDTH 9
   4. Define exponentiated numbers via the multiplication form.
         LET C = 3.01*10**6 ;. (for 3.01 times 10 to the 6)
         LET D = 1.23*10**-5 ;. (for 1.23 time 10 to the -5)
 
----------------------------------------
 
2.1.3
Operating on Numbers via the LET Command
   The LET command is frequently used for operating on numbers.
      LET A = 25
      LET B = 2/3
      LET C = ((2+3)*6)/(7-3)
      LET D = SQRT(10)
 
----------------------------------------
 
2.1.4
The Use of Numbers in Analyses
   PLOT SIN(X) FOR X = -3 .1 3
   PLOT X**2.3 FOR X = 1 1 10
   .
   SKIP 25; READ LEW.DAT
   SPECTRUM Y
   DEMODULATION FREQUENCY .3
   COMPLEX DEMODULATION FREQUENCY PLOT Y
 
----------------------------------------
 
2.2
Parameters (named scalars)
   1. Naming Conventions for Parameters
   2. Storage of Parameter Values
   3. Creating Parameters via the LET Command
   4. Creating Parameters via the READ PARAMETER Command
   5. Writing Parameters to Screen & File via the WRITE Command
   6. Neat Writing of Parameters via SET WRITE DECIMALS Command
   7. Text/Title/Label/Legend Writing of Parameters via ^
   8. Neat Text/Title/Label/Legend Writing of Parameters via ROUND()
   9. Changing Values of Parameters via the LET Command
  10. Deleting Parameters--Not Doable.
  11. Automatically-Provided Parameters--PI and INFININTY
  12. The Use of Parameters in DATAPLOT Analyses
 
----------------------------------------
 
2.2.1
Naming Conventions for Parameters
   1. Parameter names must start with an alphabetic character.
         LET A = 23
   2. Parameter names may start with any alphabetic character.
         LET OMEGA = 45
   3. Parameter names may contain any character except
         LET AB_CD = 4.65
      the mathematical and relational operators.
   4. Parameter names may be of any length but only the
      first 8 characters are internally stored.
         LET TEMPERATURE = 98.5
 
----------------------------------------
 
2.2.2
Storage of Parameter Values
   1. Parameter values are stored internally in single precision.
   2. Key intermediate calculations are in double precision.
 
----------------------------------------
 
2.2.3
Creating Parameters via the LET Command
   The LET command is frequently used for defining parameters.
      LET A = 3
      LET B = 9.24
      LET C = 3.01*10**6 ;. (for 3.01 times 10 to the 6)
      LET D = 1.23*10**-5 ;. (for 1.23 time 10 to the -5)
 
----------------------------------------
 
2.2.4
Creating Parameters via the READ PARAMETER Command
   The READ PARAMETER command is used to read parameters
   from keyboard or from file.
      READ PARAMETER A
      67.3
      WRITE A
      .
      LIST LEW.DAT FOR I = 1 1 30
      SKIP 25
      READ PARAMETER LEW.DAT. B
      WRITE B
 
----------------------------------------
 
2.2.5
Writing Parameters to Screen & File via the WRITE Command
   The WRITE command is used to write parameter values
   to screen or to file.
      LET A = 24
      LET B = SQRT(A)
      WRITE A B
      WRITE OUT. A B
 
----------------------------------------
 
2.2.6
Neat Writing of Parameters via SET WRITE DECIMALS Command
   The SET WRITE DECIMALS command is used in conjunction
   with the WRITE command to neatly write parameter values
   to screen or to file.
 
   Syntax:  SET WRITE DECIMALS    <# decimal places to retain>
   Note:    The result will have been rounded--not truncated.
 
   Example--
      LET A = 24
      LET B = SQRT(A)
      WRITE A B
      WRITE OUT. A B
      SET WRITE DECIMALS 4
      WRITE A B
      WRITE OUT. A B
      SET WRITE DECIMALS 2
      WRITE A B
      WRITE OUT. A B
 
----------------------------------------
 
2.2.7
Text/Title/Label/Legend Writing of Parameters via ^
   The ^ character
   followed immediately (that is, no space)
   by a parameter name of interest,
   followed in turn by a trailing space,
   will allow the current value of a parameter to be inserted
   into any text as in
      1. Text Output via WRITE
      2. General Text Output via TEXT
      3. Title Output on Plots via TITLE
      4. Horizontal or Vertical Label Output on Plots via ...LABEL
      5. Legend Output on Plots via LEGEND
 
----------------------------------------
 
2.2.7.1
Text Output via WRITE
      WRITE "THE VALUE OF PI IS ^PI "
      WRITE OUT. "THE VALUE OF PI IS ^PI "
 
----------------------------------------
 
2.2.7.2
General Text Output via TEXT
      LET C = 100
      LET F = 32 + (9/5)*C
      TEXT THE FAHRENHEIT BOILING TEMPERATURE IS ^F
      FONT TRIPLEX
      TEXT THE FAHRENHEIT BOILING TEMPERATURE IS ^F
 
----------------------------------------
 
2.2.7.3
Title Output on Plots via TITLE
      MULTIPLOT 2 2
      LOOP FOR K = 1 1 4
      TITLE X**^K
      PLOT X**K FOR X = 0 .1 1
      END OF LOOP
      MULTIPLOT OFF
 
----------------------------------------
 
2.2.7.4
Horizontal or Vertical Label Output on Plots via ...LABEL
      SKIP 25
      READ BERGER1.DAT X Y
      FIT Y = A+B*X
      X1LABEL THE INTERCEPT IS ^A
      X2LABEL THE SLOPE IS ^B
      X3LABEL THE RESIDUAL STANDARD DEVIATION IS ^RESSD
      CHAR X BLANK
      LINES BLANK SOLID
      PLOT Y PRED VERSUS X
 
----------------------------------------
 
2.2.7.5
Legend Output on Plots via LEGEND
      SKIP 25
      READ BERGER1.DAT X Y
      FIT Y = A+B*X
      LEGEND 1 THE INTERCEPT IS ^A
      LEGEND 2 THE SLOPE IS ^B
      LEGEND 3 THE RESIDUAL STANDARD DEVIATION IS ^RESSD
      CHAR X BLANK
      LINES BLANK SOLID
      PLOT Y PRED VERSUS X
 
----------------------------------------
 
2.2.8
Neat Text/Title/Label/Legend Writing of Parameters via ROUND()
   The ROUND(.,.) function in conjunction
   with the ^ character
   will allow the current value (to whatever neat number
   of decimal places the analyst desires) of a parameter
   to be inserted into any text as in
      1. Text Output via WRITE
      2. General Text Output via TEXT
      3. Title Output on Plots via TITLE
      4. Horizontal or Vertical Label Output on Plots via ...LABEL
      5. Legend Output on Plots via LEGEND
 
----------------------------------------
 
2.2.8.1
Text Output via WRITE
      WRITE "THE VALUE OF PI IS ^PI "
      WRITE OUT. "THE VALUE OF PI IS ^PI "
      LET PI2 = ROUND(PI,4)
      WRITE "THE VALUE OF PI IS ^PI2 "
      WRITE OUT. "THE VALUE OF PI IS ^PI2 "
 
----------------------------------------
 
2.2.8.2
General Text Output via TEXT
      LET C = 100
      LET F = 32 + (9/5)*C
      TEXT THE FAHRENHEIT BOILING TEMPERATURE IS ^F
      FONT TRIPLEX
      TEXT THE FAHRENHEIT BOILING TEMPERATURE IS ^F
      LET F = ROUND(F,2)
      TEXT THE FAHRENHEIT BOILING TEMPERATURE IS ^F
 
----------------------------------------
 
2.2.8.3
Title Output on Plots via TITLE
      MULTIPLOT 2 2
      LOOP FOR K = 1 1 4
      LET K2 = ROUND(K,1)
      TITLE X**^K2
      PLOT X**K FOR X = 0 .1 1
      END OF LOOP
      MULTIPLOT OFF
 
----------------------------------------
 
2.2.8.4
Horizontal or Vertical Label Output on Plots via ...LABEL
      SKIP 25
      READ BERGER1.DAT X Y
      FIT Y = A+B*X
      LET A = ROUND(A,3)
      LET B = ROUND(B,3)
      LET RESSD = ROUND(RESSD,4)
      X1LABEL THE INTERCEPT IS ^A
      X2LABEL THE SLOPE IS ^B
      X3LABEL THE RESIDUAL STANDARD DEVIATION IS ^RESSD
      CHAR X BLANK
      LINES BLANK SOLID
      PLOT Y PRED VERSUS X
 
----------------------------------------
 
2.2.8.5
Legend Output on Plots via LEGEND
      SKIP 25
      READ BERGER1.DAT X Y
      FIT Y = A+B*X
      LET A = ROUND(A,1)
      LET B = ROUND(B,1)
      LET RESSD = ROUND(RESSD,2)
      LEGEND 1 THE INTERCEPT IS ^A
      LEGEND 2 THE SLOPE IS ^B
      LEGEND 3 THE RESIDUAL STANDARD DEVIATION IS ^RESSD
      CHAR X BLANK
      LINES BLANK SOLID
      PLOT Y PRED VERSUS X
 
----------------------------------------
 
2.2.9
Changing Values of Parameters via the LET Command
      LET A = 3
      LET B = 7
      LET C = (A+B*2)/3.6
      WRITE A B C
      .
      LET SUM = 0
      LOOP FOR K = 1 1 6
      LET TERM = (1/2)**K
      LET SUM = SUM + TERM
      WRITE "LOOP ITERATION ^K --   SUM = ^SUM "
      END OF LOOP
 
----------------------------------------
 
2.2.10
Deleting Parameters--Not Doable.
   Once a parameter name is defined, it cannot be deleted
   and changed, say, into a variable or a function.
   The DELETE command does not work with parameters.
   Of course the value of a parameter may be changed;
   but the parameter itself may not be deleted.
 
----------------------------------------
 
2.2.11
Automatically-Provided Parameters--PI and INFININTY
   2 built-in automatic parameters--
      PI (= 3.1415926)
      INFINITY  (= max floating point value)
 
   Example of Usage--
      PLOT(2*PI*0.3*X) FOR X = 0 .1 10
 
----------------------------------------
 
2.2.12
The Use of Parameters in DATAPLOT Analyses
 
      LET AMP = 5
      LET FREQ = .25
      LET PHASE = .33
      PLOT AMP*SIN(2*PI*FREQ*X+PHASE) FOR X = 0 .1 10
      .
      SKIP 25; READ BERGER1.DAT X Y
      LINEAR FIT Y X
      WRITE A0 A1
      .
      LET GAMMA = 3
      LET Y = WEIBULL RANDOM NUMBERS FOR I = 1 1 25
      MULTIPLOT 3 3
      LOOP FOR GAMMA = 1 .5 4
      LEGEND 1 GAMMA = ^GAMMA
      WEIBULL PROBABILITY PLOT Y
      END OF LOOP
 
----------------------------------------
 
2.3
Variables (named vectors = 1-dimensional arrays)
   1. Naming Conventions for Variables
   2. Storage of Parameter Values
   3. Creating Variables via the LET Command
   4. Creating Variables via the READ PARAMETER Command
   5. Writing Variables to Screen & File via the WRITE Command
   6. Neat Writing of Variables via SET WRITE DECIMALS Command
   7. Text/Title/Label/Legend Writing of Variables via ^
   8. Neat Text/Title/Label/Legend Writing of Variables via ROUND()
   9. Changing Values of Variables via the LET Command
  10. Deleting Variables--Not Doable.
  11. Automatically-Provided Variables--PI and INFININTY
  12. The Use of Variables in DATAPLOT Analyses
 
2.3
Variables (named vectors = 1-dimensional arrays)
   Rules--
      1. Variable names must start with an alphabetic character;
      2. Variable names may start with any alphabetic character;
      3. Variable names may contain any character except
         the mathematical and relational operators.
      4. Variable names may be of any length but only the
         first 8 characters are internally stored.
      5. All values are stored internally in single precision;
      6. Key intermediate calculations are in double precision.
      7. Max number of allowable variables is
         installation-dependent.  A common maximum is
         10 variables by 2048 observations/variable.
         To determine your maximum, note the log-on banner,
         or enter the STATUS command.
     8. To change the max number of variables, use the
        DIMENSION command as in DIMENSION 20 VARIABLES .
        But if increase the default number of variables,
        then will automatically decrease the allowable
        number of observations per variable (say from 2048
     10 (say from 2048 Variables may be deleted via DELETE
     11 (say from 2048 Variables may be renamed via RENAME
     12 (say from 2048 Variables may be read in via READ and SERIAL READ
     13 (say from 2048 Variables may be written out via WRITE
      9 (say from 2048 The LET command is frequently used for defining variables
.
      8 (say from 2048 Automatically provided variables--
            After any FIT, ANOVA, SMOOTH--
               PRED (= predicted values)
               RES (= residual values = Y - predicted)
            After any plot of any kind--
               YPLOT = vertical axis trace coordinates
               XPLOT = horizontal axis trace coordinates
               X2PLOT = other horizontal axis trace coor (for 3-D)
               TAGPLOT = trace identification coded values (1, 2, 3,  (say from
2048..)
      9. The above automatically-provided variables may
         be used like any other variable in any DATAPLOT
         command; e.g., PLOT Y PRED VS X   .
 
 
 
 
 
 
 
 
 
 
 
 
 
 
 
 
 
 
 
 
 
 
 
 
 
 
 
 
 
 
 
 
 
 
 
 
 
 
 
 
 
 
 
 
 
 
 
 
 
 
 
 
 
 
 
 
 
 
 
 
 
 
 
 
 
 
 
 
 
 
 
 
 
 
 
 
 
 
 
 
 
 
 
 
 
 
 
 
 
 
 
 
 
 
 
----------  *FILES*  -------------------------
 
3
Files, Data Files, Plot Files, etc.
 
 
 
 
 
 
 
 
 
 
 
 
 
 
 
 
 
 
 
 
 
 
 
 
 
 
 
 
 
 
 
 
 
 
 
 
 
 
 
 
 
 
 
 
 
 
 
 
 
 
 
 
 
 
 
 
 
 
 
 
 
 
 
 
 
 
 
 
 
 
 
 
 
 
 
 
 
 
 
 
 
 
 
 
 
 
 
 
 
 
 
 
 
 
 
 
----------  *OPERATORS*  -------------------------
 
4
Mathematical and Relational Operators
 
 
 
 
 
 
 
 
 
 
 
 
 
 
 
 
 
 
 
 
 
 
 
 
 
 
 
 
 
 
 
 
 
 
 
 
 
 
 
 
 
 
 
 
 
 
 
 
 
 
 
 
 
 
 
 
 
 
 
 
 
 
 
 
 
 
 
 
 
 
 
 
 
 
 
 
 
 
 
 
 
 
 
 
 
 
 
 
 
 
 
 
 
 
 
 
----------  *COMMANDS*  -------------------------
 
5
Commands
   1. Core Commands
   2. Grouped Alphabetically (Go to top menu--enter T or TOP)
   3. Grouped By Function (Go to top menu--enter T or TOP)
   4. Grouped By Discipline (Go to top menu--enter T or TOP)
 
----------------------------------------
 
5.1
Core Commands--PLOT, FIT & LET
 
----------------------------------------
 
5.2
Grouped Alphabetically
   1. A-B
   2. C-D
   3. E-F
   4. G-H
   5. I-J
   6. K-L
   7. M-N
   8. O-P
   9. Q-R
  10. S-T
  11. U-V
  12. W-X
  13. Y-Z
  14. Other
 
----------------------------------------
 
5.3
Grouped by Function
   1. Graphics
   2. Word Charts/Diagrams
   3. Quantitative Analysis
   4. Plot Control/Plot Appearance
   5. Graphics Output Devices
   6. Input/Output
   7. Support
 
----------------------------------------
 
5.4
Grouped By Discipline
   1. Graphics
   2. Statistics
   3. Probability
   4. Mathematics
   5. Design of Experiment
   6. Business
   7. Other
 
 
 
 
 
 
 
 
 
 
 
 
 
 
 
 
 
 
 
 
 
 
 
 
 
 
 
 
 
 
 
 
 
 
 
 
 
 
 
 
 
 
 
 
----------  *SUBCOMMANDS UNDER LET*  -----------------------
 
6
Subcommands Under LET
   1. Statistics and Probability
   2. Mathematics/General
   3. Mathematics/Transforms
   4. Mathematics/Vector, Matrix, Complex, Polynomial, etc.
   5. Other
 
----------------------------------------
 
6.1
Statistics & Probability
   Compute Summary & Descriptive Statistics
   Sort the Data in a Variable
   Sort the Data in a Variable & Carry Along Other Variables
   Extract the Distinct Values in a Variable
   Compute Frequencies and Counts
   Carry Out a User-Defined Transformation of the Data in a Var.
   Compute Ranks (Distribution-Free Statistics)
   Integer-Code (2, 3, ..., 10 parts) the Data in a Variable
   Compute First Differences (Autocorrelation Analysis)
   Compute Cumulative Sums
   Generate Uniform & Normal Order Statistic Medians
   Generate Weibull Adjusted Ranks (Reliability Analysis)
   Create Biweight & Tricube Weights (Robust Regression)
   Compute Principle Components (Multivariate Analysis)
   Compute Variance-Covariance Matrix of a Data Matrix
   Compute Correlation Matrix of a Data matrix
   Generate Random Numbers, Permutations, Indices, and Subsamples
   Generate Bootstrap Indices and Sample
 
----------------------------------------
 
6.2
Mathematics/General
 
   Perform a "Desk Calculator" Evaluation of a Math Expression
   Create a User-Defined Parameter
   Carry Out a User-Defined Transformation of the Parameter
   Create a Variable with User-Defined Data
   Carry Out a User-Defined Transformation of the Data in a Var.
   Create User-Defined Functions
   Evaluate User-Defined Functions
 
   Compute the Sum, Product, and Integral of a Variable
   Generate First Differences from a Variable
   Generate Cumulative Sums, Products & Integrals from a Var.
   Find Roots of a Function
   Perform Numerical Differentiation & Integration of Data
   Perform Analytical Differentiation of a Function
   Perform Cubic Spline Interpolation of Data in Variables
   Perform Convolution & Deconvolution of 2 Variables
 
   Generate Prime, Fibonnacci, Logistic, Cantor Numbers
 
----------------------------------------
 
6.3
Mathematics/Transforms
   Carry Out a User-Defined Transformation of the Data in a Var.
   Perform Sine Transform of a Variable
   Perform Cosine Transform of a Variable
   Perform Laplace & Inverse Laplace Transform of a Variable
   Perform Fourier & Inverse Fourier Transform of a Variable
   Perform FFT & Inverse FFt Transform of a Variable
 
----------------------------------------
 
6.4
Mathematics/Vector, Matrix, Complex, Polynomial, etc.
   Perform Vector Operations on Variables
   Perform Matrix Operations on Matrices
   Perform Complex Number Operations on Variables
   Perform Set Operations on Variables
   Perform Logical Operations on Variables
   Perform Polynomial Arithmetic Operations on Variables
 
----------------------------------------
 
6.5
Other
   Create a Variable with User-Defined Data
   Create a Variable with a User-Defined Data Sequence
   Create a Variable with a User-Defined Data Pattern
 
   Sort the Data in a Variable
   Rank the Data in a Variable
 
   Sort the Data in a Variable & Carry Along Other Variables
   Extract the Distinct Values in a Variable
   Integer-Code (1 = min, 2 = ...) the Distinct Data in a Var.
   Carry Out a User-Defined Transformation of the Data in a Var.
 
   Extract the Distinct Values of Data in a Variable
 
 
 
 
----------  *RESERVED WORDS AND QUALIFIERS*  --------------
 
7
Reserved Words & Qualifiers
 
 
 
 
 
 
 
 
 
 
 
 
 
 
 
 
 
 
 
 
 
 
 
 
 
 
 
 
 
 
 
 
 
 
 
 
 
 
 
 
 
 
 
 
 
 
 
 
 
 
 
 
 
 
 
 
 
 
 
 
 
 
 
 
 
 
 
 
 
 
 
 
 
 
 
 
 
 
 
 
 
 
 
 
 
 
 
 
 
 
 
 
 
 
 
 
----------  *MACROS AND PROGRAMS*  ------------------------
 
8
Macros & Programs
 
 
 
 
 
 
 
 
 
 
 
 
 
 
 
 
 
 
 
 
 
 
 
 
 
 
 
 
 
 
 
 
 
 
 
 
 
 
 
 
 
 
 
 
 
 
 
 
 
 
 
 
 
 
 
 
 
 
 
 
 
 
 
 
 
 
 
 
 
 
 
 
 
 
 
 
 
 
 
 
 
 
 
 
 
 
 
 
 
 
 
 
 
 
 
 
----------  *BUILT-IN FUNCTIONS*  ------------------------
 
9
Built-in Functions
   Go to top menu--enter T or TOP
 
----------  *TEXT SUBCOMMANDS*  -------------------------
 
10
Text Subcommands
   Go to top menu--enter T or TOP
 
----------  *AUXILIARY FILES*  -------------------------
 
11
Auxiliary Files
   1. Reference Files
   2. Data Files  (Go to top menu--enter T or TOP)
   3. Map Files
   4. Design of Experiment Files
   5. Fractal Art Files
   6. Macro Files
   7. Macro Menu Files
 
----------------------------------------
 
11.1
Reference Files
   1. Chemistry/Physics
   2. Geometry Formulae
   3. Probability Functions
   4. Ascii Numeric Equivalents
   5. Dataplot Default Settings
   6. Phone & Mail
   7. Business
   8. Quality Improvement/Taguchi/Japan
   9. Latitude of U.S. Cities
   10. Weather
 
----------------------------------------
 
11.1.1
Chemistry/Physics
      ATOMS.TEX     Relative num of atoms of elements in univ
      CONSTANT.TEX  Math, physical, & engineering constants
      CONVFACT.TEX  Conversion factors between metrics/units
      SPECTRUM.TEX  Electromagnetic Spectrum (by frequency)
 
 
 
----------------------------------------
 
11.1.2
Geometry Formulae
      AREAS.TEX     Area formulae for geometric figures
      PERIM.TEX     Perimeter formulae for geometric figures
      VOLUMES.TEX   Volume formulae for geometric figures
 
 
 
----------------------------------------
 
11.1.3
Probability Functions
      PDF.TEX       Probability density functions for dist
      PPF.TEX       Percent point functions for distributions
 
 
 
----------------------------------------
 
11.1.4
Ascii Numeric Equivalents
      ASCII.TEX     ASCII numeric equivalents
 
 
 
----------------------------------------
 
11.1.5
Dataplot Default Settings
      DEFAULTS.TEX  DATAPLOT default settings
 
 
 
----------------------------------------
 
11.1.6
Phone & Mail
      AREACODE.TEX  U. S. phone area codes (alphabet. by state)
      AREACOD2.TEX  U. S. phone area codes (ordered numeric.)
      POSTCODE.TEX  U. S. postal state codes (alphabetically)
      POSTRATE.TEX  April 1988 postal rates for various weights
      PHONE.TEX     Misc. phone numbers (update locally)
 
 
 
----------------------------------------
 
11.1.7
Business
      TAX1979.TEX   1979 federal income tax rates
      TAX1987.TEX   1987 federal income tax rates
      TIMEMANA.TEX  6 techniques for time management
      CALENDAR.TEX  Calendars for the years 1988 and 1989
      FEDPAY88.TEX  U. S Federal salary scale for 1988
      MORTGAGE.TEX  Wash. D. C. 30-yr mortgage rates (Dec/87)
      JOBS.TEX      Fastest growing jobs requiring a college deg
      WORKSTAT.TEX  Top 10 computer workstations (1986)
 
 
 
----------------------------------------
 
11.1.8
Quality Improvement/Taguchi/Japan
      DEMING14.TEX  Demings's 14 "deadly diseases" (ind. qual.)
      DEMING7.TEX   Demings's 7 "dreadful diseases" (ind. qual.)
      ISHIKAW4.TEX  The 4 usual "M" components in Ishikawa Diag.
      JAPANU.TEX    Japan's 3  U's to be eliminated in work area.
      JAPANW.TEX    Japan's 5 W's (& 1 H) for component analysis
      JAPAN6.TEX    Japan's 6-point program for quality manufac.
      JAPAN3.TEX    The 3 modern concepts in Japanese industry
      KACKER.TEX    Kacker's 7 summary points of Taguchi philos.
      NEWTOOLS.TEX  The "7 New Tools" for quality control
      OLDTOOLS.TEX  The "7 Old Tools" for quality control
      QUALCOST.TEX  Japan's def and 4 comp of "quality cost"
      TAGREF.TEX    The 2 primary Tag. reference books for quality
 
 
 
----------------------------------------
 
11.1.9
Latitude of U.S. Cities
      LATITUDE.TEX  Latitude & longitude of U. S. cities
 
 
 
----------------------------------------
 
11.1.10
Weather
      HOT.TEX       10 hottest places in the U. S.
      SNOWY.TEX     10 snowiest places in the U. S.
      WINDCHIL.TEX  Windchill factor curves
 
 
 
 
----------------------------------------
 
11.2
Data Files
   Go to top menu--enter T or TOP
 
----------------------------------------
11.3
Map Files
   Dataplot has a few relatively coarse map data files.
   These will be updated in the future from high-resolution
   map files (e.g., from other government agencies).
 
      CHINA.DAT       Map coordinates for China (Map)
      NBSPART1.DAT    Map of middle of NBS campus (Map)
      TEXAS.DAT       Map coordinates for Texas (Map)
      USA.DAT         Map coordinates for USA (crude resolu.) (Map)
      USA3.DAT        Map coordinates for USA (Map)
 
      Example of usage--
      SKIP 25; READ TEXAS.DAT X Y
      PRE-SORT OFF; PLOT Y X
      GRID ON; PLOT Y X
      PLOT Y X SUBSET X 200 TO 300 SUBSET Y 0 TO 100
      SAVE 1
      FRAME OFF
      /
 
----------------------------------------
 
11.4
Design of Experiment Files
   Go to top menu--enter T or TOP
 
----------------------------------------
 
11.5
Fractal Art Files
      FRACBRAN.DAT  Fractal Data to generate a branch
      FRACCHRI.DAT  Data to gen a fractal Christmas tree
      FRACCLOU.DAT  Fractal Data to generate a cloud
      FRACFERN.DAT  Data to generate a fractal fern
      FRACFRON.DAT  Data to generate a fractal frond
      FRACGALA.DAT  Data to generate a fractal galaxy
      FRACPENT.DAT  Data to generate a fractal pentagon
      FRACSPIR.DAT  Data to generate a fractal spiral
      FRACSQUA.DAT  Data to generate a fractal square
      FRACTRIA.DAT  Data to generate a fractal triangle
 
      Example of usage (generate a triangle)--
      LIST FRACTRIA.DAT ;. (to get file reading instructions)
      SKIP 25; READ FRACBRAN.DAT Y1 TO Y7
      LINES
      CHARACTERS *
      FRAME OFF
      FRACTAL PLOT Y1 Y2 Y3 Y4 Y5 Y6 Y7
 
----------------------------------------
 
11.6
Macro Files
 
----------------------------------------
 
11.7
Menu Macro Files
   1. What is a Menu Macro File and How Use?
   2. List of Available Menu Macro Files
 
----------------------------------------
 
11.7.1.
What is a Menu Macro File and How Use?
      A menu macro file is a file containing a Dataplot
      macro (= subprogram) which, when executed, will
         1) generate menu questions to the user;
         2) take in keyboard responses to those questions;
         3) after all questions have been answered,
            carry out the prescrived operation/plot/fit/etc.
 
      To execute any of these macros, enter CALL followed by
      To execute a menu macro, enter (while in Dataplot)
      CALL followed by the desired menu macro file name
      as in
            CALL PLOTSIN.DP
      These menu macros are provided so as to illustrate how
      the user may construct/tailor similar question/answer
      menus for his/her particular application.  The simplest
      menu macro is PLOTSIN.DAT--look at this one first.
 
----------------------------------------
 
11.7.2.
List of Available Menu Macro Files
      PLOTSIN.DP    Menu macro (elementary) to plot sin function
      PLOTFUNC.DP   Menu macro to plot a function
      PLOT1VAR.DP   Menu macro (elementary) to read & plot 1 var
      PLOT2VAR.DP   Menu macro (elementary) to read & plot 2 var
      CONNDOTS.DP   Menu macro to connect dots via cross-hair
      INVMAT.DP     Menu macro to invert a matrix
      ISHIKAWA.DP   Menu macro to gen an Ishikawa diagram
      MULTTEXT.DP   Menu macro to position text via cross-hair
      NORMHIST.DP   Menu macro to gen hist + normal density
      PARETO.DP     Menu macro to generate a Pareto plot
      PIECHART.DP   Menu macro to generate labelled pie chart
      RANDSAMP.DP   Menu macro to gen stratified random sample
      SIMPMETH.DP   Menu macro to compute simplex sol (lin prog)
      SORT.DP       Menu macro to sort alpha list via priorities
      SUM.DP        Menu macro to sum a list of numbers
      TTEST.DP      Menu macro to perfrom t test on 2 var
      WORDCHAV.DP   Menu macro to gen centered vert wordchart
      WORDCHAH.DP   Menu macro to gen centered hor wordchart
      23CUBE.DP     Menu macro to gen a 2**3 fact exp des cube
      3DPLOT.DP     Menu macro to gen a 3dplot with cube frame
 
 
 
 
 
 
 
 
 
 
 
 
 
 
 
 
 
 
 
 
 
 
 
 
 
 
 
 
 
 
 
 
 
 
 
 
 
 
 
 
 
 
 
 
 
 
 
 
 
 
 
 
 
 
 
 
 
 
 
 
 
 
 
 
 
 
 
 
 
 
 
 
 
 
 
 
 
 
 
 
 
 
 
 
 
 
 
 
 
 
 
 
 
 
 
 
----------  *TUTORIAL*  ------------------
 
12
Tutorial
   Go to top menu--enter T or TOP
 
----------------------------------------
 
----------  *SHORTCUTS*  -------------------------
 
13
Shortcuts/Saving Keystrokes
 
----------------------------------------
 
 
 
 
 
 
 
 
 
 
 
 
 
 
 
 
 
 
 
 
 
 
 
 
 
 
 
 
 
 
 
 
 
 
 
 
 
 
 
 
 
 
 
 
 
 
 
 
 
 
 
 
 
 
 
 
 
 
 
 
 
 
 
 
 
 
 
 
 
 
 
 
 
 
 
 
 
 
 
 
 
 
 
 
 
 
----------  *HARDWARE*  -------------------------
 
14
Hardware, CPU's, Operating Systems, etc.
 
----------------------------------------
 
 
 
 
 
 
 
 
 
 
 
 
 
 
 
 
 
 
 
 
 
 
 
 
 
 
 
 
 
 
 
 
 
 
 
 
 
 
 
 
 
 
 
 
 
 
 
 
 
 
 
 
 
 
 
 
 
 
 
 
 
 
 
 
 
 
 
 
 
 
 
 
 
 
 
 
 
 
 
 
 
 
 
 
 
 
 
 
 
 
 
 
 
 
 
----------  *GRAPHICS DRIVERS*  ------------------------
 
15
Graphics Drivers/Devices, Postscript, X-Windows, etc.
 
 
----------------------------------------
 
 
 
 
 
 
 
 
 
 
 
 
 
 
 
 
 
 
 
 
 
 
 
 
 
 
 
 
 
 
 
 
 
 
 
 
 
 
 
 
 
 
 
 
 
 
 
 
 
 
 
 
 
 
 
 
 
 
 
 
 
 
 
 
 
 
 
 
 
 
 
 
 
 
 
 
 
 
 
 
 
 
 
 
 
 
 
 
 
 
 
 
 
 
----------  *INSTALLATION AND TAILORING HINTS*  -----------
 
16
Installation & Tailoring Hints
 
----------------------------------------
 
 
 
 
 
 
 
 
 
 
 
 
 
 
 
 
 
 
 
 
 
 
 
 
 
 
 
 
 
 
 
 
 
 
 
 
 
 
 
 
 
 
 
 
 
 
 
 
 
 
 
 
 
 
 
 
 
 
 
 
 
 
 
 
 
 
 
 
 
 
 
 
 
 
 
 
 
 
 
 
 
 
 
 
 
 
 
 
 
 
 
 
 
 
 
----------  *DEFAULT SETTINGS*  -------------------------
 
17
Default Settings
 
----------------------------------------
 
 
 
 
 
 
 
 
 
 
 
 
 
 
 
 
 
 
 
 
 
 
 
 
 
 
 
 
 
 
 
 
 
 
 
 
 
 
 
 
 
 
 
 
 
 
 
 
 
 
 
 
 
 
 
 
 
 
 
 
 
 
 
 
 
 
 
 
 
 
 
 
 
 
 
 
 
 
 
 
 
 
 
 
 
 
 
 
 
 
 
 
 
 
 
----------  *HELP AND DOCUMENTATION*  --------------
 
18
On-Line/Off-Line Help & Documentation
 
--------------------------------------------------
