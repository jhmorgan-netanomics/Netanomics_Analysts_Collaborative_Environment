38.                          number of     sections below
15.                          section number for the   A     section
  100.    OVERVIEW           first line number of OVERVIEW section
  400.    ABBREVIATIONS      first line number of ABBREVIATIONS section
  500.    DEFAULTS           first line number of DEFAULTS section
  600.    EXAMPLES           first line number of EXAMPLES section
  800.    FEATURES           first line number of FEATURES section
  900.    HOSTS              first line number of HOST section
 1000.    IMPLEMENTATION     first line number of IMPLEMENTATION section
 1200.    LANGUAGE           first line number of LANGUAGE section
 1300.    MODES              first line number of OUTPUT DEVICES section
 1400.    OUTPUT DEVICES     first line number of OUTPUT DEVICES section
 1500.    PORTABILITY        first line number of PORTABILITY section
 1600.    SUMMARY            first line number of SUMMARY section
 1800.    TERMINALS          first line number of TERMINALS section
 1900.    TUTORIAL           first line number of TUTORIAL section
 2200.    A                  first line number of A section
 2400.    B                  first line number of B section
 2500.    C                  first line number of C section
 2800.    D                  first line number of D section
 3200.    E                  first line number of E section
 3500.    F                  first line number of F section
 3600.    G                  first line number of G section
 3700.    H                  first line number of H section
 3900.    I                  first line number of I section
 4200.    J                  first line number of J section
 4300.    K                  first line number of K section
 4400.    L                  first line number of L section
 4900.    M                  first line number of M section
 5000.    N                  first line number of N section
 5100.    O                  first line number of O section
 5200.    P                  first line number of P section
 5400.    Q                  first line number of Q section
 5500.    R                  first line number of R section
 5600.    S                  first line number of S section
 5900.    T                  first line number of T section
 6100.    U                  first line number of U section
 6200.    V                  first line number of V section
 6300.    W                  first line number of W section
 6400.    X                  first line number of X section
 6500.    Y                  first line number of Y section
 6600.    Z                  first line number of Z section
 
 
 
 
 
 
 
 
Updated--July 1992 (for merge with DATAPLOT)
 
 
 
 
 
 
 
 
 
 
 
 
 
 
 
 
 
 
 
 
 
 
 
 
 
 
 
 
 
 
 
 
 
 
 
 
 
 
 
 
 
 
 
 
 
 
 
----------------------------------------------------------
-------------------------  *OVERVIEW*  -------------------
----------------------------------------------------------
 
OVERVIEW
 
OVERVIEW
 
   EDIT/FED is an English-syntax, portable line editor
   for general-purpose text-editing, program-writing,
   and word-processing.
 
   Examples of EDIT/FED commands--
 
      EXIT       = Exit from EDIT/FED
      ABORT      = Abort from EDIT/FED (no changes made)
      P          = Print current line
      P 7        = Print 7 lines (current + 6)
      P 20 30    = Print lines 20 to 30
      23         = Go to (and print) line 23
      <CR>       = Go to (and print) next line
      N 4        = Go to (and print) 4th next line
      U 5        = Go up 5 lines (and print it)
      L ABC      = Locate next occurrance of ABC
      C /ABC/DE/ = Change ABC to DE on current line
      D          = Delete current line
      I ABC      = Insert ABC as a line after current line
      INPUT      = Go into input (= insert) mode
      EDIT       = Go back to edit mode (the default)
 
   Additional information about EDIT/FED is generated
   via the HELP command.  The format of the
   HELP command is
 
      HELP   [command or topic]
 
 
Summary (1-line) of EDIT/FED commands                 Example
- - - - - - - - - - - - - - - - - - - - - - - - - - - - - - -
 
ABORT     Abort out of editor (no changes made)       AB
 
ADD       Insert a text file after the current line   ADD ABC.TEX
 
BOTTOM    Go to bottom (= last line + 1) of file      B
 
CALL      Execute an editor macro (subprogram)        CALL ABC.TEX
 
CENTER    Center lines at specified column            SET CENTER 25
                                                      CENTER
 
CHANGE    Change old string to new string             C /KAT/CAT/
                                                      C /KAT/CAT/ 20
 
CA        Change all occurances of string on line     CA /KAT/CAT/
                                                      CA /KAT/CAT/ 20
 
COPY      Copy pre-marked text block to a file        SET COPY 20 30
                                                      COPY ABC TEX
 
CUT       Trim text off end of line                   CUT CAT
 
DCOPY     Copy (& delete) pre-marked block to file    SET COPY 20 30
                                                      DCOPY ABC TEX
 
DELETE    Delete lines of text                        D
                                                      D 10
                                                      D 5 14
 
DHOLD     Delete a line but hold (= save) it          DHOLD
                                                      DHOLD 2
                                                      DHOLD 10
 
DI        Delete and insert a line                    DI XXX
 
DUP       Duplicate (= insert) held lines             DUP
          (from HOLD and DHOLD)                       DUP 2
                                                      DUP 10
                                                      DUP 1 5
                                                      DUP 2 7
 
DTL       Delete lines until locate a string          DTL --
 
EDIT      Switch from input mode to edit mode         EDIT
 
EXECUTE   Execute a held line                         X
 
EXTEND    Append a string onto a line                 EXT  SUBSET
 
EXIT      Exit out of editor (make changes permanent) EXIT
 
EXEM      Append a string to multiple lines           EXT  SUBSET 10
 
GO        Go to a line                                GO 25
                                                      or 25
 
HELP      Display on-line help information            HELP
                                                      HELP PRINT
                                                      HELP CHANGE
 
HOLD      Hold (= saves) a line                       HOLD
                                                      HOLD 2
                                                      HOLD 10
 
INDENT    Indent lines at specified column            SET INDENT 3
                                                      INDENT
                                                      INDENT 10
 
INPUT     Switch from edit mode to input mode         INPUT
 
INSERT    Insert a line after the current line        I PLOT Y X
 
LAST      Go to the last line of the file             LAST
 
LOCATE    Locate next line with a string              L PLOT
 
LOCATE    Locate all succeeding lines with a string   LA PLOT
 
LI        Locate all lines with string & and
          insert a line after                         LI PLOT XXX
 
LIB       Locate all lines with string & and
          insert a line before                        LIB PLOT XXX
 
 
LIST      List an external file                       LIST ABC.TEX
 
NEAT      Make paragraph neat                         SET NEAT 1 50
                                                      NEAT
 
NEXT      Go to the next line                         N
                                                      or <CR>
                                                      N 10
 
PRINT     Print lines                                 P
                                                      P 10
                                                      P 5 14
 
PA        Print (current +) all remaining lines       PA
 
PN        Print lines (starting with next line)       PN
                                                      PN 20
 
PP        Print a page of text                        PP
 
PPAR      Print a paragraph of text                   PPAR
 
PTL       Print lines until locate a string           PTL --
 
SET CENT  Set (= mark) column for CENTER              SET CENTER 25
 
SET CHAN  Set (= mark) columns for CHANGE             SET CHANGE 7 20
 
SET COPY  Set (= mark) columns for COPY               SET COPY 70 90
 
SET LOCA  Set (= mark) columns for LOCATE             SET LOCATE 5 7
 
SET NEAT  Set (= mark) columns for NEAT               SET NEAT 1 50
 
SET SHIF  Set (= mark) columns for SHIFT              SET SHIFT 55
 
SET TRUN  Set (= mark) columns for TRUNCATE           SET TRUNCATE 55
 
SC1       Set (= mark) first line for COPY            SC1
 
SC2       Set (= mark) last  line for COPY            SC2
 
SHIFT     Shift text left or right on lines           SET SHIFT 5
                                                      SHIFT
                                                      SHIFT 20
 
SHOW      Display settings of switches and limit      SHOW
 
SPLIT     Split remainder of a line onto next line    SPLIT CAT
 
STAT      Display settings of switches and limit      STAT
 
TOP       Go to the top (= line 1 - 1) of the file    T
 
TRUNCATE  Truncate lines at a specified column        SET TRUN 55
                                                      TRUNCATE
                                                      TRUNCATE 300
 
UNDO      Undo all changes (from previous TOP)        UNDO
 
UP        Go up to previous line                      UP
                                                      UP 10
 
 
   Available commands:
 
      ABORT     ADD
      BOTTOM
      CALL      CENTER    CHANGE    CA        COPY      CUT
      DELETE    DHOLD     DI        DUP       DTL
      EDIT      EXECUTE   EXTEND    EXIT      EXEM
      FIND
      GO
      HELP      HOLD
      INDENT    INPUT     INSERT
      LAST      LOCATE    LA      LI        LC        LIST
                LOBL      LABL
      NEAT      NEXT
      PRINT     PA        PN        PP        PPAR      PTL
      SET CENT  SET CHAN  SET COPY  SET LOCA
      SET NEAT  SET SHIF  SET TRUN  SC1       SC2
      TOP       TRUNCATE
      UNDO      UP
 
   Available topics:
 
      Abbreviations (for commands)
      Defaults
      Examples
      Features
      Hosts
      Implementation
      Language Extendability
      Modes (Input/Edit)
      Output Devices
      Portability
      Summary (of commands)
      Terminals
      Tutorial
 
   You can  abbreviate any topic name, although
   ambiguous abbreviations will result in only
   the first match being displayed.
 
   Examples--HELP PRINT
             HELP PTL
             HELP LOCATE
 
             HELP SUMMARY
             HELP MODES
             HELP FEATURES
 
----------------------------------------------------------
 
 
 
 
 
 
 
 
 
 
 
 
 
 
 
 
 
 
 
 
 
 
 
 
 
 
 
 
 
 
 
 
 
 
 
 
 
 
 
 
 
 
 
 
 
 
 
 
 
 
 
 
 
 
 
 
 
 
 
 
 
 
----------------------------------------------------------
-------------------------  *ABBREVIATIONS*  -------------------
----------------------------------------------------------
 
ABBREVIATIONS
 
Abbreviations for EDIT/FED commands--
 
   ABORT        --AB           ABO          ABO+anything
   ABORT & RERUN--ABRR
   ADD          --AD
 
   BOTTOM       --B            B+anything
 
   CALL         --CAL          CAL+anything
   CENTER       --CE           CE+anything
   CHANGE       --C            CH           CH+anything
   CHANGE ALL   --CA
   COPY         --CO           COP
   CUT          --CU
 
   DELETE       --D            DE           DE+anything
   DEL. & HOLD  --DH           DH+anything
   DEL. & INSERT--DI
   DUPLICATE    --DU           DU+anything
   DEL. TILL LOC--DTL
 
   EDIT         --EDIT
   EXECUTE      --EXE          EXEC         EXEC+anything
   EXTEND       --EXT          EXT+anything
   EXIT         --E            EX           EXI
   EXIT & RERUN --EXRR
 
   FIND         --F            FI           FI+anything
   FORMAT       --FO           FO+anything
 
   GO           --G
 
   HELP         --HE           HE+anything
   HOLD         --H            HO           HO+anything
 
   INDENT       --IND          IND+ANYTHING
   INPUT        --INPUT
   INSERT       --I            IN           INS          INS+anything
 
   LAST         --LAS          LAS+anything
   LOCATE       --L            LO           LO+anything
   LOCATE ALL   --LA
   LOC. BL. LINE--LOBL
   LOC. ALL B L --LABL
   LOC. & CALL  --LC
   LOC. & INSERT--LI
   LIST         --LIS
 
   MOVE         --M            M+anything
 
   NEXT         --N            N+anything
 
   PRINT        --P            PR           PR+anything
   PRINT ALL    --PA
   PRINT NEXT   --PN
   PRINT PAGE   --PP
   PRINT PAR.   --PPAR
   PR. TILL LOC.--PTL
 
   SET          --S            SE
   SHIFT        --SHI          SHI+anything
   SHOW         --SH           SHO
   SPLIT        --SP           SP+anything
 
   TOP          --T            TO
   TRUNCATE     --TR           TRU          TRU+anything
   TRANSLATE    --TRA          TRA+anything
 
----------------------------------------------------------
 
 
 
 
 
 
 
 
 
 
 
 
 
 
 
 
 
 
 
 
 
 
 
 
 
----------------------------------------------------------
-------------------------  *DEFAULTS*  -------------------
----------------------------------------------------------
 
DEFAULTS
 
DEFAULTS
 
   Maximum number of characters   per line   =     132
   Maximum number of lines      in workspace =  25,000
   Maximum number of characters in workspace = 500,000
 
   Column limits for CHANGE command          = 1 to 132
   Column limits for PRINT  command          = 1 to 132
 
   Mask character      = *
   Feedback            = ON
   Auto line numbering = ON
   Trace               = OFF
   Justification       = LEFT
 
   Host                = VAX 11/780
   Link                = Any link and baud rate
   Terminal            = Any alphanumeric terminal
   Terminal Rows       = Any number of rows
   Terminal Columns    = 60+ characters (else wrap-around)
 
----------------------------------------------------------
 
 
 
 
 
 
 
 
 
 
 
 
 
 
 
 
 
 
 
 
 
 
 
 
 
 
 
 
 
 
 
 
 
 
 
 
 
 
 
 
 
 
 
 
 
 
 
 
 
 
 
 
 
 
 
 
 
 
 
 
 
 
 
 
 
 
 
 
 
 
 
 
----------------------------------------------------------
-------------------------  *EXAMPLES*  -------------------
----------------------------------------------------------
 
EXAMPLES
 
No examples available as of yet.
 
----------------------------------------------------------
 
 
 
 
 
 
 
 
 
 
 
 
 
 
 
 
 
 
 
 
 
 
 
 
 
 
 
 
 
 
 
 
 
 
 
 
 
 
 
 
 
 
 
 
 
 
 
 
 
 
 
 
 
 
 
 
 
 
 
 
 
 
 
 
 
 
 
 
 
 
 
 
 
 
 
 
 
 
 
 
 
 
 
 
 
 
 
 
 
 
 
 
 
 
 
 
 
 
 
 
 
 
 
 
 
 
 
 
 
 
 
 
 
 
 
 
 
 
 
 
 
 
 
 
 
 
 
 
 
 
 
 
 
 
 
 
 
 
 
 
 
 
 
 
 
 
 
 
 
 
 
 
 
 
 
 
 
 
 
 
 
 
 
 
 
 
 
 
 
 
 
 
 
 
 
 
 
 
 
 
 
 
 
 
 
 
 
 
 
 
 
----------------------------------------------------------
-------------------------  *FEATURES*  -------------------
----------------------------------------------------------
 
FEATURES
 
No description available as of yet.
 
----------------------------------------------------------
 
 
 
 
 
 
 
 
 
 
 
 
 
 
 
 
 
 
 
 
 
 
 
 
 
 
 
 
 
 
 
 
 
 
 
 
 
 
 
 
 
 
 
 
 
 
 
 
 
 
 
 
 
 
 
 
 
 
 
 
 
 
 
 
 
 
 
 
 
 
 
 
 
 
 
 
 
 
 
 
 
 
 
 
 
 
 
 
 
 
 
----------------------------------------------------------
-------------------------  *HOSTS*  -------------------
----------------------------------------------------------
 
HOSTS
 
HOSTS
 
   The undelying code for EDIT/FED is portable
   FORTRAN 77; thus EDIT/FED will execute
   on any host which
 
      1) has a FORTRAN 77 compiler; and
 
      2) has sufficient main/disk
         memory for EDIT/FED's internal
         workspace (25,000 lines and
         500,000 characters).
 
   To simplify maintenance and local
   modifications to EDIT/FED, EDIT/FED's internal
   COMMON arrays have been isolated into
   the file EDCOMM.INC, and are included
   in the source code via FORTRAN's
   INCLUDE command.  Inasmuch as INCLUDE
   is not part of the FORTRAN 77 standard,
   a few (but not many) FORTRAN compilers will
   balk at this.  If your compiler complains,
   then the implementor must edit EDCOMM.INC
   into each subroutine (a straighforward
   operation) prior to compilation.  Aside from
   INCLUDE, there are no other deviations from
   the FORTRAN 77 standard--the code will
   compile cleanly.
 
   EDIT/FED will run on virutally any major
   main/mini/micro-computer.  The default
   host for EDIT/FED is the VAX 11/7XX under
   the VMS operating system.  EDIT/FED will
   run under a variety of other operating
   systems including UNIX.
 
   Known EDIT/FED implementations to date
   (November, 1985) include--
 
      VAX 11/785 under VMS
 
 
----------------------------------------------------------
 
 
 
 
 
 
 
 
 
 
 
 
 
 
 
 
 
 
 
 
 
 
 
 
 
 
 
 
 
 
 
 
 
 
 
 
 
 
 
 
 
 
 
 
 
 
 
 
 
 
 
----------------------------------------------------------
-------------------------  *IMPLEMENTATION*  -------------------
----------------------------------------------------------
 
IMPLEMENTATION
 
IMPLEMENTATION
 
   EDIT/FED is written in portable FORTRAN 77
   and so it may be implemented on any
   host computer which
 
      1) has a FORTRAN 77 compiler; and
 
      2) has sufficient main/disk
         memory for EDIT/FED's internal
         workspace (25,000 lines and
         500,000 characters).
 
   The underlying code for EDIT/FED consists of
   about 100 FORTRAN subroutines.  These
   subroutine names all have    ED   as the
   first 2 characters, as in EDMAIN, EDSEAR,
   EDSET, EDSHOW, EDCHAN, etc.
 
   Step 1 in the implementation process is
   to read the EDIT/FED tape onto your computer.
   This ascii tape was created on a
   VAX 11/785 and has the following
   specifications--
 
      Format          = Ascii
      Type            = 9-track
      Density         = 1600 Bpi
      Parity          = ???
      Labelling       = Unlabelled
      Record Type     = Fixed length
      Block Type      = Fixed length
      Record Size     = 80   characters per record
      Block  Size     = 36   records    per block
                      = 2880 characters per block
      Number of Files = 5
 
   File 1 consists of subroutines and files
   which are the most frequent candidates
   for host-dependent changes (if such changes
   need be made).  File 1 consists of
 
      EDMAIN.FOR    EDIT/FED's main routine.
      EDINIT.FOR    EDIT/FED's initialization subroutine.
      EDOPEN.FOR    EDIT/FED's file-opening subroutine.
      EDCOMM.INC    EDIT/FED's COMMON arrays
 
   File 2 consists of the remaining EDIT/FED
   subroutines ordered alphabetically.
   File 3 consists of EDIT/FED's on-line
   documentation file (EDHELP.TEX).
   File 4 consists of EDIT/FED's test
   runstream file (EDTEST.TEX).
   File 5 consists of the VAX 11/785
   runstream that would read the tape,
   compile the subroutines, link the
   subroutines, and run a simple test
   edit session.
 
   Step 2 in the implementation process is
   to compile the code in files 1 and 2.
   To simplify maintenance and local
   modifications to EDIT/FED, EDIT/FED's internal
   COMMON arrays have been isolated into
   the file EDCOMM.INC, and are includable
   in the source code via FORTRAN's
   INCLUDE command.  Inasmuch as INCLUDE
   is not part of the FORTRAN 77 standard,
   a few (but not many) FORTRAN compilers will
   balk at this.  If your compiler complains,
   then the implementor must edit EDCOMM.INC
   into each subroutine (a straighforward
   operation) prior to compilation.  Aside from
   INCLUDE, there are no other deviations from
   the FORTRAN 77 standard--the code will
   compile cleanly.
 
   Step 3 is to LINK the code and determine
   if it will fit on your computer.  The current
   EDIT/FED workspace settings have been set at
   25,000 lines and 500,000 characters.
   These settings are defined in file EDCOMM.INC.
   Link the code and see if your computer
   willkl accept the default sizes.  If
   yes, then the implementation is done;
   if no, then scale down the 25,000 and
   500,000 settings in EDCOMM.INC to
   acceptable sizes (e.g., 1000 and 50,000),
   recompile all of the subroutines, and
   relink.
 
   EDIT/FED will run on virutally any major
   main/mini/micro-computer.  The default
   host for EDIT/FED is the VAX 11/7XX under
   the VMS operating system.  EDIT/FED will
   run under a variety of other operating
   systems including UNIX.
 
   Known EDIT/FED implementations to date
   (November, 1985) include--
 
      VAX 11/785 under VMS
      IBM-PC     under DOS
 
----------------------------------------------------------
 
 
 
 
 
 
 
 
 
 
 
 
 
 
 
 
 
 
 
 
 
 
 
 
 
 
 
 
 
 
 
 
 
 
 
 
 
 
 
 
 
 
 
 
 
 
 
 
 
 
 
 
 
 
 
 
 
 
 
 
 
 
 
 
 
 
 
 
 
 
 
 
 
 
 
 
 
 
 
 
 
 
 
 
 
 
 
 
 
----------------------------------------------------------
-------------------------  *LANGUAGE EXTENSIONS*  ---------------
----------------------------------------------------------
 
No description available as of yet.
 
----------------------------------------------------------
 
 
 
 
 
 
 
 
 
 
 
 
 
 
 
 
 
 
 
 
 
 
 
 
 
 
 
 
 
 
 
 
 
 
 
 
 
 
 
 
 
 
 
 
 
 
 
 
 
 
 
 
 
 
 
 
 
 
 
 
 
 
 
 
 
 
 
 
 
 
 
 
 
 
 
 
 
 
 
 
 
 
 
 
 
 
 
 
 
 
 
 
 
----------------------------------------------------------
-------------------------  *MODES*  -------------------
----------------------------------------------------------
 
MODES
 
MODES
 
   When you are using EDIT/FED to edit a file,
   EDIT/FED will be in one of 2 modes--
 
      1) edit  mode (the default mode); or
      2) input mode.
 
   When EDIT/FED is in    edit mode,
   then all of the usual editor commands
   (such as PRINT, CHANGE, DELETE, LOCATE,
   NEXT, COPY, MOVE, ADD, LIST, EXIT, etc.)
   are at the disposal of the analyst.
   The     edit mode    is the most common
   mode of operation; most changing and updating
   of files is done in    edit mode.
 
   The advantage of    edit mode   is that
   all of the editor commands may be used to
   make the desired changes in the file.
   The disadvantage of    edit mode is that
   extra keystrokes are required for the entry
   of large amounts of text.
 
   When EDIT/FED is in    input mode,
   then "what you type" is "what goes into
   the file".  Input mode   is the most
   common way of entering large amounts
   of text into a file.
 
   The advantage of    input mode   is that
   large amounts of text may be efficiently
   entered.   The disadvantage of    input mode
   is that the analyst has no immediate access
   to editor commands (such as PRINT, DELETE,
   and so forth).
 
   If you are in   edit mode   , and you want
   to go into   input mode   , there is only
   one way to do it--by entering
 
      INPUT
 
   If you are in   input mode   , and you want
   to go into   edit mode   , there is only
   one way to do it--by entering
 
      EDIT
 
----------------------------------------------------------
 
 
 
 
 
 
 
 
 
 
 
 
 
 
 
 
 
 
 
 
 
 
 
 
 
 
 
 
 
 
 
 
 
 
 
 
 
 
 
 
 
 
 
 
----------------------------------------------------------
-------------------------  *OUTPUT DEVICES*  -------------------
----------------------------------------------------------
 
OUTPUT DEVICES
 
 
No description available as of yet.
 
----------------------------------------------------------
 
 
 
 
 
 
 
 
 
 
 
 
 
 
 
 
 
 
 
 
 
 
 
 
 
 
 
 
 
 
 
 
 
 
 
 
 
 
 
 
 
 
 
 
 
 
 
 
 
 
 
 
 
 
 
 
 
 
 
 
 
 
 
 
 
 
 
 
 
 
 
 
 
 
 
 
 
 
 
 
 
 
 
 
 
 
 
 
 
 
----------------------------------------------------------
-------------------------  *PORTABILITY*  -------------------
----------------------------------------------------------
 
PORTABILITY
 
 
No description available as of yet.
 
----------------------------------------------------------
 
 
 
 
 
 
 
 
 
 
 
 
 
 
 
 
 
 
 
 
 
 
 
 
 
 
 
 
 
 
 
 
 
 
 
 
 
 
 
 
 
 
 
 
 
 
 
 
 
 
 
 
 
 
 
 
 
 
 
 
 
 
 
 
 
 
 
 
 
 
 
 
 
 
 
 
 
 
 
 
 
 
 
 
 
 
 
 
 
 
----------------------------------------------------------
-------------------------  *SUMMARY*  -------------------
----------------------------------------------------------
 
SUMMARY
 
Summary (1-line) of EDIT/FED commands                 Example
- - - - - - - - - - - - - - - - - - - - - - - - - - - - - - -
 
ABORT     Abort out of editor (no changes made)       AB
 
ADD       Insert a text file after the current line   ADD ABC.TEX
 
BOTTOM    Go to bottom (= last line + 1) of file      B
 
CALL      Execute an editor macro (subprogram)        CALL ABC.TEX
 
CENTER    Center lines at specified column            SET CENTER 25
                                                      CENTER
 
CHANGE    Change old string to new string             C /KAT/CAT/
                                                      C /KAT/CAT/ 20
 
CA        Change all occurances of string on line     CA /KAT/CAT/
                                                      CA /KAT/CAT/ 20
 
COPY      Copy pre-marked text block to a file        SET COPY 20 30
                                                      COPY ABC TEX
 
CUT       Trim text off end of line                   CUT CAT
 
DCOPY     Copy (& delete) pre-marked block to file    SET COPY 20 30
                                                      DCOPY ABC TEX
 
DELETE    Delete lines of text                        D
                                                      D 10
                                                      D 5 14
 
DHOLD     Delete a line but hold (= save) it          DHOLD
                                                      DHOLD 2
                                                      DHOLD 10
 
DI        Delete and insert a line                    DI XXX
 
DUP       Duplicate (= insert) held lines             DUP
          (from HOLD and DHOLD)                       DUP 2
                                                      DUP 10
                                                      DUP 1 5
                                                      DUP 2 7
 
DTL       Delete lines until locate a string          DTL --
 
EDIT      Switch from input mode to edit mode         EDIT
 
EXECUTE   Execute a held line                         X
 
EXTEND    Append a string onto a line                 EXT  SUBSET
 
EXIT      Exit out of editor (make changes permanent) EXIT
 
EXEM      Append a string to multiple lines           EXT  SUBSET 10
 
GO        Go to a line                                GO 25
                                                      or 25
 
HELP      Display on-line help information            HELP
                                                      HELP PRINT
                                                      HELP CHANGE
 
HOLD      Hold (= saves) a line                       HOLD
                                                      HOLD 2
                                                      HOLD 10
 
INDENT    Indent lines at specified column            SET INDENT 3
                                                      INDENT
                                                      INDENT 10
 
INPUT     Switch from edit mode to input mode         INPUT
 
INSERT    Insert a line after the current line        I PLOT Y X
 
LAST      Go to the last line of the file             LAST
 
LOCATE    Locate next line with a string              L PLOT
 
LOCATE    Locate all succeeding lines with a string   LA PLOT
 
LI        Locate all lines with string & and
          insert a line after                         LI PLOT XXX
 
LIB       Locate all lines with string & and
          insert a line before                        LIB PLOT XXX
 
 
LIST      List an external file                       LIST ABC.TEX
 
NEAT      Make paragraph neat                         SET NEAT 1 50
                                                      NEAT
 
NEXT      Go to the next line                         N
                                                      or <CR>
                                                      N 10
 
PRINT     Print lines                                 P
                                                      P 10
                                                      P 5 14
 
PA        Print (current +) all remaining lines       PA
 
PN        Print lines (starting with next line)       PN
                                                      PN 20
 
PP        Print a page of text                        PP
 
PPAR      Print a paragraph of text                   PPAR
 
PTL       Print lines until locate a string           PTL --
 
SET CENT  Set (= mark) column for CENTER              SET CENTER 25
 
SET CHAN  Set (= mark) columns for CHANGE             SET CHANGE 7 20
 
SET COPY  Set (= mark) columns for COPY               SET COPY 70 90
 
SET LOCA  Set (= mark) columns for LOCATE             SET LOCATE 5 7
 
SET NEAT  Set (= mark) columns for NEAT               SET NEAT 1 50
 
SET SHIF  Set (= mark) columns for SHIFT              SET SHIFT 55
 
SET TRUN  Set (= mark) columns for TRUNCATE           SET TRUNCATE 55
 
SC1       Set (= mark) first line for COPY            SC1
 
SC2       Set (= mark) last  line for COPY            SC2
 
SHIFT     Shift text left or right on lines           SET SHIFT 5
                                                      SHIFT
                                                      SHIFT 20
 
SHOW      Display settings of switches and limit      SHOW
 
SPLIT     Split remainder of a line onto next line    SPLIT CAT
 
STAT      Display settings of switches and limit      STAT
 
TOP       Go to the top (= line 1 - 1) of the file    T
 
TRUNCATE  Truncate lines at a specified column        SET TRUN 55
                                                      TRUNCATE
                                                      TRUNCATE 300
 
UNDO      Undo all changes (from previous TOP)        UNDO
 
UP        Go up to previous line                      UP
                                                      UP 10
 
----------------------------------------------------------
 
 
 
 
 
 
 
 
 
 
 
 
 
 
 
 
 
 
 
 
 
 
 
 
 
 
 
 
 
 
 
 
 
 
 
 
 
 
 
 
 
 
-------------------------------------------------------------
-------------------------  *TERMINALS*  -------------------
----------------------------------------------------------
 
TERMINALS
 
No description available as of yet.
 
----------------------------------------------------------
 
 
 
 
 
 
 
 
 
 
 
 
 
 
 
 
 
 
 
 
 
 
 
 
 
 
 
 
 
 
 
 
 
 
 
 
 
 
 
 
 
 
 
 
 
 
 
 
 
 
 
 
 
 
 
 
 
 
 
 
 
 
 
 
 
 
 
 
 
 
 
 
 
 
 
 
 
 
 
 
 
 
 
 
 
 
 
 
 
 
 
----------------------------------------------------------
-------------------------  *TUTORIAL*  -------------------
----------------------------------------------------------
 
TUTORIAL
 
No tutorial available as of yet.
 
----------------------------------------------------------
 
      Carriage Return     .        **ST      #       <Name>
 
      SCALE
      SET NUMBER
SET PROMPT ON/OFF
SET INFINITY
SET MASK
SUMMARY
WORD PROCESSING
SET FONT
SET HW
SET JUSTIFICATION
MOVE ON PAGE
DRAW LINE
SET LASER ON/OFF
EXIT-&-RERUN
ABORT-&-RERUN
SET NEAT ON/OFF
Generate Format Statements
 
   Note--The SET command and the SHOW command
          have many sub-entries.
 
   Enter INSTRUCTIONS for more detailed information
   on how to use HELP.
 
   Enter HINTS if you are not sure of the name
   of the command or topic for which you need help.
 
 
 
 
 
 
 
 
 
 
 
 
 
 
 
 
 
 
 
 
 
 
 
 
 
 
 
 
 
 
 
 
 
 
 
 
 
 
 
 
 
 
 
 
 
 
 
 
 
 
 
 
 
 
 
 
 
 
 
 
 
 
 
 
 
 
 
 
 
 
 
 
 
 
 
 
 
 
 
 
 
 
 
 
 
 
 
 
 
 
 
 
 
 
 
 
 
 
 
 
 
 
 
 
 
 
 
 
 
 
 
 
 
 
 
 
 
 
 
 
 
 
 
 
 
 
 
 
 
 
 
 
 
 
 
 
 
 
 
 
 
 
 
 
 
 
 
 
 
 
 
 
 
 
 
 
 
 
 
 
 
 
 
 
 
 
 
 
 
 
 
 
 
 
 
 
 
 
 
 
 
 
 
 
 
 
 
 
 
 
 
 
 
 
 
 
 
 
 
 
 
 
 
 
 
 
 
 
 
 
 
 
 
 
 
 
 
 
 
 
 
 
 
 
 
 
 
 
 
 
 
 
 
 
 
 
 
 
 
 
 
 
 
 
 
 
 
 
 
 
 
 
 
 
 
 
 
 
 
 
 
 
----------------------------------------------------------
-------------------------  *A*  --------------------------
----------------------------------------------------------
 
ABORT
 
Command --ABORT
 
Purpose --Exits out of the file being edited
          with all edit changes being ignored;
          that is, the file is left in its
          original state.
 
Format  --ABORT
 
Synonyms--AB
 
Examples--ABORT
          AB
 
Related Commands--
          EXIT            = Exits from editing session.
          ABRR            = Aborts and reruns editor.
          EXRR            = Exits and reruns editor.
 
----------------------------------------------------------
 
 
ABRR
 
Command --ABRR (Abort and Rerun)
 
Purpose --Aborts out of the file being edited
          (with all edit changes being ignored--
          that is, the file is left in its
          original state); further, you are
          immediately placed back in the editor
          with an inquiry as to what new
          file you would like to edit.
 
Format  --ABRR
 
Synonyms--None.
 
Examples--ABRR
 
Related Commands--
          ABORT           = Aborts out of editing session.
          EXIT            = Exits  out of editing session.
          EXRR            = Exits  out and reruns editor.
 
----------------------------------------------------------
 
ADD
 
Command --ADD
 
Purpose --Adds (= inserts) a block of text or
          an external file.  The insertion is done
          immediately after the current line.
 
Format  --The ADD command has 2 forms--
 
       1--ADD
          This no-argument form adds the
          last block of text that was
          saved via the COPY command
          with no arguments.
 
       2--ADD    [file name]
          This 1-argument form adds the
          specified external file.
 
Note    --For both forms, after the ADD command
          is executed, the "current line" is
          updated to the last line of the added text.
 
Note    --If the number of lines added is
          less than 10, then all lines
          will be printed on the screen
          as they are being added.
        --If the number of lines added
          is greater than 10, then only the
          first and last added lines will
          be printed.
        --If no printing is desired, then
          the FEEDBACK OFF command should
          be entered beforehand.
 
Synonyms--AD
 
Examples--ADD
          ADD REPORT.TEX
 
Related Commands--
          COPY            = Copies a block of text.
          MOVE            = Moves a block of text.
          SET FEEDBACK    = Allows/suppresses feedback.
 
----------------------------------------------------------
 
 
 
 
 
 
 
 
 
 
 
 
 
 
 
 
 
 
 
 
 
 
 
 
 
 
 
 
 
 
 
 
 
 
 
 
 
 
 
 
 
 
 
 
 
 
 
 
 
 
 
 
 
 
 
 
 
 
 
 
 
 
 
 
 
 
 
 
 
 
 
 
 
 
 
 
 
 
 
 
 
 
 
 
 
 
 
 
 
 
 
 
 
 
 
 
 
 
 
 
----------------------------------------------------------
-------------------------  *B*  --------------------------
----------------------------------------------------------
 
BOTTOM
 
Command --BOTTOM
 
Purpose --Goes to the (imaginary) line immediately
          after the last line of the file.
 
Note    --The "current line" setting is updated
          to the     total number of lines + 1.
 
        --[BOTTOM]    appears on the screen.
 
Format  --BOTTOM
 
Synonyms--BOT
 
Examples--BOT
          BOTTOM
 
Related Commands--
          TOP             = Goes before line 1.
          LAST            = Goes to last line.
          FIRST           = Goes to first line.
          GO              = Goes to any line.
          #               = Goes to any line.
          SET NUMBER      = Allows/suppresses line numbers.
          SET PROMPT      = Allows/suppresses prompt & line number.
 
----------------------------------------------------------
 
 
 
 
 
 
 
 
 
 
 
 
 
 
 
 
 
 
 
 
 
 
 
 
 
 
 
 
 
 
 
 
 
 
 
 
 
 
 
 
 
 
 
 
 
 
 
 
 
 
 
 
 
 
 
 
 
 
 
 
 
 
 
 
 
 
 
----------------------------------------------------------
-------------------------  *C*  --------------------------
----------------------------------------------------------
 
CALL
 
Command --CALL
 
Purpose --Calls (= executes) an edit macro
          (= subroutine = procedure)
          residing in the specified file.
 
Format  --The CALL command has 2 forms--
 
       1--CALL
          This no-argument form re-executes
          the last macro that had been executed.
 
       2--CALL    [file name]
          This 1-argument form executes the
          macro in the specified file.
 
Note    --If one is unsure of the contents
          of a macro file, then use of the LIST
          command will allow one to pre-view
          the file (without actually executing
          the contents of the file).
 
Synonyms--CA
          CAL
 
Examples--CALL
          CALL REPORT.TEX
 
Related Commands--
          LIST            = Lists (= prints on the screen)
                            a file (but does not execute it).
 
----------------------------------------------------------
 
CENTER
 
Command --CENTER
 
Purpose --Centers existing text in the specified range
          of lines so that each line is centered about
          a pre-specified target column.
 
Format  --The CENTER command has 3 forms--
 
       1--CENTER
 
          This no-argument form centers the current
          line (only) so that the resulting text
          is centered about the target column.
 
       2--CENTER     [number of lines]
 
          This 1-argument form centers each line
          for the specified number of lines.  The
          centering starts with the current line.
          Each line is centered about the target column.
 
       3--CENTER   [first line]   [last line]
 
          This 2-argument form centers each line
          for the specified range of lines; that is,
          centering starts with the   [first line]
          and proceeds through the   [last line].
          Each line is centered about the target column.
 
Note    --For all 3 cases, the    [target column]
          is pre-specified by prior use of the
          SET CENTER command, as in
 
             SET CENTER 30
             CENTER 100 120
 
          which would set the [target column] to 30
          and then center lines 100 through 120.
 
        --The centering operation takes the existing
          text sting on each line, determines the first
          non-blank character, determines the last non-blank
          character, and re-positions the entire text string
          so that mid-string is at the    [target column].
 
        --The CENTER command is useful for the
          centering of titles and the like.
 
        --The CENTER command operates on existing text,
          text which is already in the specified line.
          It has no effect on yet-to-be-entered text.
 
        --For all 3 forms, after the CENTER command
          has been executed, the "current line" is
          updated to be the last line centered.
 
Synonyms--CE
          CEN
 
Examples--SET CENTER 30    Specifies centering column as 30.
          CENTER           Centers current line about column 30.
          CENTER 5         Centers next 5 lines about column 30
                           (starting with current line).
          CENTER 10 15     Centers lines 10 to 15 about column 30.
 
Related Commands--
          SET CENTER      = Specifies the centering column.
          INDENT          = Indents text on lines.
          TRUNCATE        = Truncates text on lines.
          SHIFT           = Shifts text on lines.
          SET TAB         = Specifies column for auto-tabbing.
          PRINT           = Prints lines.
          SET FEEDBACK    = Allows/suppresses feedback.
 
----------------------------------------------------------
 
COPY
 
Command --COPY
 
Purpose --Copies a block of text
             1) to the editor's temporary copy file; or
             2) to an external file; or
             3) to another location in the file.
 
Format  --The COPY command has 3 forms--
 
       1--COPY
          This no-argument form copies the
          previously-specified block of text
          to the editor's temporary copy file.
          The previous-specification is done via the
          SET BEGIN, SET END, or SET COPY  commands).
          The "current line" setting remains unchanged.
 
       2--COPY    [file name]
          This 1-argument form copies the
          previously-specified block of text
          to the specified file.
          The previous-specification is done via the
          SET BEGIN, SET END, or SET COPY  commands).
          The "current line" setting remains unchanged.
 
       3--COPY   [start line]   [stop line]   [target line]
          This 3-argument form copies the block of text
          from the    start line   through the    stop line
          and inserts this block immediately after
          the    target line.
          The "current line" setting is updated to be
          the last line of the copied block in its
          new location.
 
Note    --The copy does not destroy the original version
          of the text block; it is a copy (= duplicate);
          it is not a move and delete.
 
Note    --If the number of lines copied is
          less than 10, then all lines
          will be printed on the screen
          as they are being copied.
        --If the number of lines copied
          is greater than 10, then only the
          first and last copied lines will
          be printed.
        --If no printing is desired, then
          the FEEDBACK OFF command should
          be entered beforehand.
 
Synonyms--CO
 
Examples--3 ways to copy the text from
          lines 50 to 60 and insert it
          immediately after line 20--
 
          1) SET COPY  50 60
             COPY
             20
             ADD
 
          2) SET COPY  50 60
             COPY YOURFILE.TEX
             20
             ADD YOURFILE.TEX
 
          3) COPY 50 60 20
 
Related Commands--
          SET BEGIN       = Defines first line of block.
          SET END         = Defines last line of block.
          SET COPY        = Defines first & last lines of block.
          ADD             = Adds (= inserts) a block of text.
          MOVE            = Moves a block of text.
          SET FEEDBACK    = Allows/suppresses printing.
 
----------------------------------------------------------
 
CUT
 
Command --CUT
 
Purpose --Deletes all characters on the current
          line beyond (and including) the
          specified string.
 
Format  --CUT   [string]
 
          Deletes all characters on the current
          line beyond (and including) [string].
 
Note    --If multiple occurrances of the string
          are on the line, then the cutting will
          be done from the last (= right-most) such
          occurrance.  Thus if the original line is
             ABC DEF GHI ABC DEF GHI
          Then the command
             CUT DEF
          will result in
             ABC DEF GHI ABC
          The commands
             CUT DE
          and
             CUT D
          will yield the same result.
 
        --The CUT command has no effect
          on the "current line" setting--it
          remains unchanged.
 
Synonyms--None.
 
Examples--CUT ABC          Cuts ABC and beyond on current line.
        --CUT )            Cuts ) and beyond on current line.
 
Related Commands--
          EXTEND          = Appends string to end of line.
          SPLIT           = Splits string onto next line.
          SET TAB         = Sets auto-tab column.
          SET INDENT      = Sets indentation column.
          SET CENTER      = Sets centering column.
          SET TRUNCATE    = Sets truncation column.
          INDENT          = Indents lines of text.
          CENTER          = Centers lines of text.
          TRUNCATE        = Truncates lines of text.
          DELETE          = Deletes k lines (start with current).
          PRINT           = Prints k lines (start with current).
          SET FEEDBACK    = Allows/suppresses printing.
          SET NUMBER      = Allows/suppresses line numbers.
          SET PROMPT      = Allows/suppr. prompt & line num.
 
----------------------------------------------------------
 
 
 
 
 
 
 
 
 
 
 
 
 
 
 
 
 
 
 
 
 
 
 
 
 
 
 
 
 
 
 
 
 
 
 
 
 
 
 
 
 
 
 
 
 
 
 
 
----------------------------------------------------------
-------------------------  *D*  --------------------------
----------------------------------------------------------
 
DELETE
 
Command --DELETE
 
Purpose --Deletes specified lines.
 
Format  --The DELETE command has 3 forms--
 
       1--DELETE
 
          This no-argument form deletes the
          current line only.
          After the deletion is completed, the
          "current line" setting is left unchanged,
          and a    [DELETED LINE]   message is
          printed out.
 
       2--DELETE [number of lines]
 
          This 1 argument form deletes the specified
          [number of lines].  The deleting starts with
          the current line.
          After the deletion is completed, the
          "current line" setting is left unchanged,
          and a    [DELETED LINE]   message is
          printed out.
 
       3--DELETE [first line] [last line]
 
          This 2-argument form deletes the specified
          range of lines, that is deletes from the
          [first line]     through the     [last line].
          After the deletion is completed, the
          "current line" setting is updated to
          the first line of the deleted block,
          and a    [DELETED LINE]   message is
          printed out.
 
Synonyms--D
 
Examples--D                    Delete current line.
          D 10                 Delete next 10 lines.
                               (starting with current line).
          D 30 50              Delete lines 30 to 50.
 
Related Commands--
          DTL             = Deletes till locate a string.
          DHOLD           = Deletes and holds (saves) a line.
          PRINT           = Prints lines.
          SET FEEDBACK    = Allows/suppresses printing.
          SET NUMBER      = Turns line numbers on/off.
          SET PROMPT      = Turns prompt on/off.
 
----------------------------------------------------------
 
DHOLD
 
Command --DHOLD (Delete and Hold)
 
Purpose --Deletes and holds (= saves = copies)
          the current line.
 
Format  --The DHOLD command has 2 forms--
 
       1--DHOLD
 
          This no-argument form deletes and holds
          (= saves) the current line.  The line
          is saved in the editor's temporary line-
          storage location #1 (there are 10 such
          line-storage locations).  The   DHOLD
          command with no arguments is equivalent to
          DHOLD 1   .
 
       2--DHOLD    [storage number]
 
          This 1-argument form deletes and holds
          (= saves) the current line.  The line
          is saved in the editor's temporary line-
          storage location specified by the
          [storage number].  There are 10 such line-
          storage locations and so     [storage number]
          must be an integer from 1 to 10.
 
Note    --After the DHOLD is executed, the
          "current line" setting is left unchanged,
          and a    [DELETE LINE]   message is
          printed out.
 
        --To dump a held line, use the DUP command.
          The DHOLD and DUP commands are complementary
          in the sense that DHOLD deletes and
          holds a desired line, while DUP inserts
          the held line at some other desired
          location in the text.  Thus DUP 3 will insert
          the contents of the editor's line-storage
          location #3 immediately after whatever line
          the editor is currently positioned at.
          Thus an example of editor code to
             go to line 50
             delete and hold that line
             then go to line 20
             and insert the held line
          is
             50
             DHOLD
             20
             DUP
 
Synonyms--DH
          HD
 
Examples--DH      Delete and hold line in ed. temp. location #1.
          DH 1    Delete and hold line in ed. temp. location #1.
          DH 5    Delete and hold line in ed. temp. location #5.
 
Related Commands--
          DUP             = Duplicates (= insert)
                            a held line.
          HOLD            = Holds a line
                            (but do not delete it).
          DELETE          = Deletes lines.
          DTL             = Deletes lines till locate a string.
          MOVE            = Moves (= delete & copy)
                            a block of lines.
          PRINT           = Prints lines.
          SET FEEDBACK    = Allows/suppresses printing.
 
----------------------------------------------------------
 
DUPLICATE
 
Command --DUPLICATE
 
Purpose --Duplicates (= inserts) a held line.
 
Format  --The DUP command has 3 forms--
 
       1--DUP
 
          This no-argument form duplicates once
          (= inserts) the held line residing in the
          editor's temporary line-storage location #1.
          The insertion is done immediately after the
          current line.  The DUPLICATE command with
          no arguments is equivalent to   DUPLICATE 1
          and    DUPLICATE 1 1    (see forms 2 and 3 below).
 
       2--DUPLICATE    [storage number]
 
          This 1-argument form duplicates once
          (= inserts) the held line residing in the
          editor's temporary line-storage location
          specified by   [storage number].  Since there
          are 10 such line-storage locations available,
          then   [storage number]   must be an integer
          from 1 to 10.  The DUPLICATE command with
          1 argument (e.g., DUPLICATE 7) is equivalent to
          the 2-arguemnt DUPLICATE command with second
          argument = 1 (e.g., DUPLICATE 7 1);
          see the 2-argument form below.
          The insertion is done immediately after the
          current line.
 
       3--DUPLICATE  [storage number]  [number of duplicates]
 
          This 2-argument form duplicates
          [number of duplicates] times
          the held line residing in the
          editor's temporary line-storage location
          specified by   [storage number].  Since there
          are 10 such line-storage locations available,
          then   [storage number]   must be an integer
          from 1 to 10.  The duplication is done
          immediately after the current line.
 
Note    --If the number of lines duplicated is
          less than 10, then all lines
          will be printed on the screen
          as they are being inserted.
        --If the number of lines duplicated
          is greater than 10, then only the
          first and last inserted lines will
          be printed.
        --If no printing is desired, then
          the FEEDBACK OFF command should
          be entered beforehand.
 
        --After the DUPLICATE is executed, the
          "current line" setting is updated to the last
          line printed.
 
        --To hold a desired line, use the HOLD command
          (= Hold but do not delete) or the
          DHOLD (= Hold and then delete) commands.
          The HOLD/DHOLD and DUPLICATE commands are complementary
          in the sense that HOLD/DHOLD holds
          a desired line, while DUPLICATE inserts
          the held line at some other desired
          location in the text.  Thus DUPLICATE 3 will insert
          the contents of the editor's line-storage
          location #3 immediately after whatever line
          the editor is currently positioned at.
          Thus an example of editor code to
             go to line 50
             hold (= save) that line
             then go to line 20
             and insert the held line
          is
             50
             HOLD
             20
             DUPLICATE
 
Synonyms--DU
 
Examples--DUPLICATE        Duplicate line 1 once.
          DUPLICATE 7      Duplicate line 7 once.
          DUPLICATE 7 50   Duplicate line 7 50 times.
 
Related Commands--
          HOLD            = Holds a line
                            (but does not delete it).
          DHOLD           = Holds a line
                            (but also deletes it).
          EXECUTE         = Executes a held line.
          INSERT          = Inserts a line.
          ADD             = Inserts a file.
          CALL            = Executes a file.
          COPY            = Copies a block of lines.
          MOVE            = Moves (= deletes and copies)
                            a block of lines.
          PRINT           = Prints lines.
          SET FEEDBACK    = Allows/suppresses feedback.
          SET PRINTING    = Allows/suppresses printing.
 
----------------------------------------------------------
 
DTL
 
Command --DTL (Delete till Locate)
 
Purpose --Locates the next line on which
          the specified string occurs, and
          deletes all lines from the current line
          down to the line with the located string.
 
Format  --The DTL command has 2 forms--
 
       1--DTL
 
          This no-argument form locates the
          next line on which the previously-
          specified string occurs, and
          deletes all lines from the current line
          down to the line with the located string.
          The "previously-specified string" is
          the string that was explicitly used in a
          previous LOCATE, DTL, DTL, or LAI command.
          Thus the no-argument form of the DTL command
          "remembers" previous located strings
          and uses them if the analyst so desires.
 
       2--DTL     [string]
 
          This 1-argument form locates the
          next line on which the specified
          string occurs and  deletes all lines
          from the current line down to the line
          with the located string.
 
Note    --The deleting starts on the current line;
          but the search for the string
          does not start until the next line.
          If the string is found anywhere on
          some line, the search is terminated,
          and the lines are deleted.
 
        --Each line is searched its full width
          (the default column limits for the search
          are columns 1 to 132).
          To restrict the search to a certain range  of
          columns (e.g., columns 7 through 72), then
          use the SET LOCATE command before using
          the DTL command, as in
             SET LOCATE 7 72
             DTL ABC
          which will delete the current line
          and every line down to the first
          line containing ABC in columns
          7 through 72.
 
        --If the specified string is not located,
          then the current line and all subsequent
          lines are deleted.
 
        --For both forms of the DTL command, after the
          command has been executed, the "current line"
          setting is updated to the last line that was
          printed.
 
 
Synonyms--None.
 
Examples--DTL ABC              Delete current line down
                               to the first line containing
                               the string     ABC
        --DTL 1000 CONTINUE    Delete current line down
                               to the first line containing
                               the string     1000 CONTINUE
 
Related Commands--
          SET LOCATE      = Sets column limits for
                            LOCATE, DTL, DTL, and
                            LAI commands.
          LOCATE          = Locates a string.
          PTL             = Print-till-locate =
                            prints from current line down
                            to line with located string.
          LAI             = Locate-all-and-insert =
                            locates all lines with a string
                            and inserts new lines with a second
                            string.
          PRINT           = Prints k lines (start with current).
          SET FEEDBACK    = Allows/suppresses printing.
          SET NUMBER      = Allows/suppresses line numbers.
          SET PROMPT      = Allows/suppr. prompt & line num.
 
----------------------------------------------------------
 
 
 
 
 
 
 
 
 
 
 
 
 
 
 
 
 
 
 
 
 
 
 
 
 
 
 
 
 
 
 
 
 
 
 
 
 
 
 
 
 
 
 
 
 
 
 
 
 
 
 
 
 
 
 
 
 
 
 
 
 
 
 
 
 
 
 
----------------------------------------------------------
-------------------------  *E*  --------------------------
----------------------------------------------------------
 
EDIT
 
Command --EDIT
 
Purpose --Place the editor into    edit mode
          so that all subsequent lines entered
          from the terminal will be interpreted
          as editor commands, and will not be
          placed into the file as raw text.
 
Format  --EDIT
 
Note    --The editor has 2 modes--
 
             1) edit  mode (default mode)
             2) input mode
 
        --When the editor is in    edit mode,
          then all of the editor commands are
          at the disposal of the analyst.
          The     edit mode    is the most common
          mode of operation; most changing and updating
          of files is done in    edit mode.
          The advantage of    edit mode   is that
          all of the editor commands may be used to
          make the desired changes in the file.
          The disadvantage of    edit mode is that
          extra keystrokes are required for the entry
          of large amounts of text.
 
        --When the editor is in    input mode,
          then "what you type" is "what goes into
          the file".  Input mode   is the most
          common way of entering large amounts
          of text into a file.
          The advantage of    input mode   is that
          large amounts of text may be efficiently
          entered.   The disadvantage of    input mode
          is that the analyst has no immediate access
          to editor commands (such as PRINT, DELETE,
          and so forth).
 
        --To go from edit mode to input mode, enter
 
             INPUT
 
        --To go from input mode to edit mode, enter
 
             EDIT
 
Synonyms--None
 
Examples--EDIT
 
Related Commands--
          INPUT           = Goes from edit mode to input mode.
 
----------------------------------------------------------
 
EXECUTE
 
Command --EXECUTE
 
Purpose --Executes a held line.
 
Format  --The EXECUTE command has 3 forms--
 
       1--EXECUTE
 
          This no-argument form executes once
          the held line residing in the
          editor's temporary line-storage location #1.
          The EXECUTE command with no arguments
          is equivalent to   EXECUTE 1    and
          EXECUTE 1 1    (see forms 2 and 3 below).
 
       2--EXECUTE    [storage number]
 
          This 1-argument form executes once
          the held line residing in the
          editor's temporary line-storage location
          specified by   [storage number].  Since there
          are 10 such line-storage locations available,
          then   [storage number]   must be an integer
          from 1 to 10.  The EXECUTE command with
          1 argument (e.g., EXECUTE 7) is equivalent to
          the 2-arguemnt EXECUTE command with second
          argument = 1 (e.g., EXECUTE 7 1);
          see the 2-argument form below.
 
       3--EXECUTE  [storage number]  [number of duplicates]
 
          This 2-argument form executes
          [number of duplicates] times
          the held line residing in the
          editor's temporary line-storage location
          specified by   [storage number].  Since there
          are 10 such line-storage locations available,
          then   [storage number]   must be an integer
          from 1 to 10.
 
Note    --The EXECUTE command is analagous to
          the CALL command in that they both
          execute editor instructions.  EXECUTE
          differs from CALL in that EXECUTE executes
          a single held instruction line while
          CALL executes a file-full of instructions.
 
        --The EXECUTE command is analagous to
          the DUPLICATE command in that they both
          operate on held lines (held via
          HOLD/DHOLD).  EXECUTE differs from
          DUPLICATE in that EXECUTE executes
          a held line while DUPLICATE merely
          duplicates (= inserts) a held line.
 
        --Whether or not printing results
          from the commands being executed,
          depends on the actual commands
          being executed and on the setting
          of the FEEDBACK and PRINTING
          switches.  If no printing is desired,
          then such switches should be turned off
          ahead of time by prior entry of the
          SET FEEDBACK OFF and SET PRINTING OFF
          commands.
 
        --After the EXECUTE is executed, the
          "current line" setting is updated
          depending on what instructions
          have been executed.
 
        --To hold a desired line, use the HOLD command
          (= Hold but do not delete) or the
          DHOLD (= Hold and then delete) commands.
          The HOLD/DHOLD and EXECUTE commands are complementary
          in the sense that HOLD/DHOLD holds
          a desired line, while EXECUTE executes
          the held line .  Thus EXECUTE 3 will execute
          the contents of the editor's line-storage
          location #3 immediately after whatever line
          the editor is currently positioned at.
          Thus an example of editor code to
             go to line 50
             hold (= save) that line
             then execute that line
          is
             50
             HOLD
             EXECUTE
 
Synonyms--EXE
 
Examples--EXECUTE        Execute held line 1 once.
          EXECUTE 7      Execute held line 7 once.
          EXECUTE 7 50   Execute held line 7 50 times.
 
Related Commands--
          HOLD            = Holds a line
                            (but do not delete it).
          DHOLD           = Holds a line
                            (but also delete it).
          DUPLICATE       = Inserts a held line.
          CALL            = Executes a file.
          ADD             = Inserts a file.
          INSERT          = Inserts a line.
          PRINT           = Prints lines.
          SET FEEDBACK    = Allows/suppresses feedback.
          SET PRINTING    = Allows/suppresses printing.
 
----------------------------------------------------------
 
EXTEND
 
Command --EXTEND
 
Purpose --Attaches a string at the end
          of the current line.
 
Format  --EXTEND   [string]
 
          Attaches [string] at the end
          of the current line.
 
Note    --All editor commands must have at least 1 space
          between the command and the succeeding
          arguments of the command line.  For
          most editor commands; extra spaces
          are not significant.  For the EXTEND
          command, however, such extra spaces are
          significant--they become part of the string
          to be appended.  For example, if the
          original line is
             ABC DEF GHI
          and the command
             EXTEND JKL MNO PQR
          is entered (note only one space
          after the EXTEND command), then
          the resulting line is
             ABC DEF GHIJKL MNO PQR
          The    J   of the new string abutts
          immediately up against the    I
          of the old string.  In some cases
          this is desirable; in others, it may
          not be.  To cause the new string to
          have a space after the old string,
          one should enter instead
             EXTEND  JKL MNO PQR
          (Note the 2 spaces after the EXTEND
          command).  The net result will be
             ABC DEF GHI JKL MNO PQR
          as desired.
 
        --The EXTEND command has not effect
          on the "current line" setting--it
          remains unchanged.
 
Synonyms--EXT
 
Examples--EXT ABC          Appends ABC to current line.
        --EXT )            Appends ) to current line.
 
Related Commands--
          CUT             = Cuts string from end of line.
          SPLIT           = Splits string onto next line.
          SET TAB         = Sets auto-tab column.
          SET INDENT      = Sets indentation column.
          SET CENTER      = Sets centering column.
          SET TRUNCATE    = Sets truncation column.
          INDENT          = Indents lines of text.
          CENTER          = Centers lines of text.
          TRUNCATE        = Truncates lines of text.
          DELETE          = Deletes k lines (start with current).
          PRINT           = Prints k lines (start with current).
          SET FEEDBACK    = Allows/suppresses printing.
          SET NUMBER      = Allows/suppresses line numbers.
          SET PROMPT      = Allows/suppr. prompt & line num.
 
----------------------------------------------------------
 
EXIT
 
Command --EXIT
 
Purpose --Exits out of the file being edited
          with all edit changes being made permanent.
 
Format  --EXIT
 
Synonyms--EX
 
Examples--EXIT
          EX
 
Related Commands--
          ABORT           = Aborts from editing session.
          EXRR            = Exits and reruns editor.
          ABRR            = Aborts and reruns editor.
 
----------------------------------------------------------
 
EXRR
 
Command --EXRR (Exit and Rerun)
 
Purpose --Exits out of the file being edited
          (with all edit changes being made permanent);
          and then immediately places you back
          in the editor with an inquiry as to
          what new file you would like to edit.
 
Format  --EXRR
 
Synonyms--None.
 
Examples--EXRR
 
Related Commands--
          EXIT            = Exits  out of editing session.
          ABORT           = Aborts out of editing session.
          ABRR            = Aborts and reruns editor.
 
----------------------------------------------------------
 
 
 
 
 
 
 
 
 
 
 
 
 
----------------------------------------------------------
-------------------------  *F*  --------------------------
----------------------------------------------------------
 
 
 
 
 
 
 
 
 
 
 
 
 
 
 
 
 
 
 
 
 
 
 
 
 
 
 
 
 
 
 
 
 
 
 
 
 
 
 
 
 
 
 
 
 
 
 
 
 
 
 
 
 
 
 
 
 
 
 
 
 
 
 
 
 
 
 
 
 
 
 
 
 
 
 
 
 
 
 
 
 
 
 
 
 
 
 
 
 
 
 
 
 
 
 
 
 
----------------------------------------------------------
-------------------------  *G*  --------------------------
----------------------------------------------------------
 
GO
 
Command --GO
 
Purpose --Goes to the specified line and
          prints the line out (on the screen).
 
Format  --GO     [line number]
 
Note    --After the GO command has been executed,
          the "current line" setting is updated
          to the specified line.
 
Note    --The GO command is a heavily used command;
          the GO command spelling is, however, rarely
          used.  It is more common to simply enter the
          [line number]     without the prefix GO.
          Thus to go to line 17, it is rare to use
             GO 17
          It is more common to use simply
             17
 
Synonyms--G
          line number
 
Examples--GO 140            Go to to line 140.
          G 140             Go to to line 140.
          140               Go to to line 140.
 
Related Commands--
          UP              = Goes up k lines.
          NEXT            = Goes down k lines.
          TOP             = Goes before line 1.
          FIRST           = Goes to first line.
          LAST            = Goes to last line.
          BOTTOM          = Goes after last line.
          GO              = Goes to any line.
          #               = Goes to any line.
          PRINT           = Prints k lines (start with current).
          PN              = Prints k lines (start with next).
          SET NUMBER      = Allows/suppresses line numbers.
          SET PROMPT      = Allows/suppr. prompt & line num.
 
----------------------------------------------------------
 
 
 
 
 
 
 
 
 
 
 
 
 
 
 
 
 
 
 
 
 
 
 
 
 
 
 
 
 
 
 
 
 
 
 
 
 
 
 
 
 
 
 
 
 
 
 
 
 
 
 
 
----------------------------------------------------------
-------------------------  *H*  --------------------------
----------------------------------------------------------
 
HELP
 
Command --HELP
 
Purpose --Invokes the editor facility to display
          information about an editor command or topic.
 
Format  --The HELP command has 2 forms--
 
       1--HELP
 
          This no-argument form displays general
          information about the HELP facility, and
          also displays a list of editor commands
          and topics for which further HELP
          information is available.
 
       2--HELP [command or topic]
 
          The 1 argument form displays specific
          format and usage information about the
          specified command or the specified topic.
          You can abbreviate any command or topic
          name, although ambiguous abbreviations will
          result in all matches being displayed.
 
Note    --For both forms, the "current line" status
          is left unchanged.
 
Synonyms--HE
 
Examples--HELP          Display general editor information.
          HELP SET      Display information about SET command.
          HELP SHOW     Display information about SHOW command.
          HELP CHANGE   Display information about CHANGE command.
          HELP C        Display information about all C commands.
          HELP LIMITS   Display information about various limits.
 
Related Commands--
          SET             = Assigns settings
                            of underlying variables.
          SHOW            = Displays current settings
                            of underlying variables.
          SET PRINTER     = Activates/deactivates hardcopy.
 
----------------------------------------------------------
 
HOLD
 
Command --HOLD
 
Purpose --Holds (= saves = copies) the current line.
 
Format  --The HOLD command has 2 forms--
 
       1--HOLD
 
          This no-argument form holds
          (= saves) the current line.  The line
          is saved in the editor's temporary line-
          storage location #1 (there are 10 such
          line-storage locations).  The   HOLD
          command with no arguments is equivalent to
          HOLD 1   .
 
       2--HOLD    [storage number]
 
          This 1-argument form holds
          (= saves) the current line.  The line
          is saved in the editor's temporary line-
          storage location specified by the
          [storage number].  There are 10 such line-
          storage locations and so     [storage number]
          must be an integer from 1 to 10.
 
Note    --After the HOLD is executed, the
          "current line" setting is left unchanged.
 
        --To dump a held line, use the DUP command.
          The HOLD and DUP commands are complementary
          in the sense that HOLD holds
          a desired line, while DUP inserts
          the held line at some other desired
          location in the text.  Thus DUP 3 will insert
          the contents of the editor's line-storage
          location #3 immediately after whatever line
          the editor is currently positioned at.
          Thus an example of editor code to
             go to line 50
             hold (= save) that line
             then go to line 20
             and insert the held line
          is
             50
             HOLD
             20
             DUP
 
Synonyms--H
          HO
 
Examples--H      Hold line in ed. temp. location #1.
          H 1    Hold line in ed. temp. location #1.
          H 5    Hold line in ed. temp. location #5.
 
Related Commands--
          DUP             = Duplicates (= inserts)
                            a held line.
          DHOLD           = Holds a line
                            (but also delete it).
          DELETE          = Deletes lines.
          DTL             = Deletes lines till locate string.
          MOVE            = Moves (= delete & copy)
                            a block of lines.
          PRINT           = Prints lines.
          SET FEEDBACK    = Allows/suppresses printing.
 
----------------------------------------------------------
 
 
 
 
 
 
 
 
 
 
 
 
 
 
 
 
 
 
 
 
 
 
 
 
 
 
 
 
 
 
 
 
 
 
 
 
 
 
 
 
 
 
 
 
 
 
 
 
 
 
 
 
 
 
 
 
 
 
 
 
 
 
 
 
 
 
 
 
 
 
 
 
 
 
 
 
 
 
----------------------------------------------------------
-------------------------  *I*  --------------------------
----------------------------------------------------------
 
INDENT
 
Command --INDENT
 
Purpose --Indents existing text in the specified range
          of lines so that each line is indented to
          the specified target column.
 
Format  --The INDENT command has 3 forms--
 
       1--INDENT
 
          This no-argument form indents the current
          line (only) so that the resulting text
          is indented to the target column.
 
       2--INDENT     [number of lines]
 
          This 1-argument form indents each line
          for the specified number of lines.  The
          indenting starts with the current line.
          Each line is indented to the target column.
 
       3--INDENT   [first line]   [last line]
 
          This 2-argument form indents each line
          for the specified range of lines; that is,
          indenting starts with the   [first line]
          and proceeds through the   [last line].
          Each line is indented to the target column.
 
Note    --For all 3 cases, the    [target column]
          is pre-specified by prior use of the
          SET INDENT command, as in
             SET INDENT 10
             INDENT 100 120
          which would set the [target column] to 10
          and then indent lines 100 through 120.
 
        --The indenting operation takes the existing
          text sting on each line, determines the first
          non-blank character, and re-positions the entire
          text string so that the first non-blank
          character is at the     [target column].
 
        --INDENT is similar to SHIFT in that
          both result in text lines being shifted.
          INDENT differs from SHIFT in that
          INDENT assures that the resulting text
          will start in the specified column,
          wheras SHIFT merely does a straightforwward
          translation (left or right).
 
        --The INDENT command is useful for the
          indenting of lists and the like.
 
        --The INDENT command operates on existing text,
          text which is already in the specified line.
          It has no effect on yet-to-be-entered text.
 
        --For all 3 forms, after the INDENT command
          has been executed, the "current line" is
          updated to be the last line indented.
 
Synonyms--IND
 
Examples--SET INDENT 6     Specifies indentation column as 6.
          INDENT           Indent current line to column 6.
          INDENT 10        Indent next 10 lines to column 6
                           (starting with current line).
          INDENT 10 15     Indent lines 10 to 15 to column 6.
 
Related Commands--
          SET INDENT      = Specifies indentation column.
          SET SHIFT       = Specifies shift amount.
          SHIFT           = Shifts text on lines.
          CENTER          = Centers text on lines.
          TRUNCATE        = Truncates text on lines.
          SET TAB         = Specifies auto-tabbing column.
          PRINT           = Prints lines.
          SET FEEDBACK    = Allows/suppresses screen printing.
 
----------------------------------------------------------
 
INPUT
 
Command --INPUT
 
Purpose --Place the editor into    input mode
          so that all subsequent lines entered
          from the terminal will be placed in the
          file as raw text, and will not be
          interpreted as editor commands.
 
Format  --INPUT
 
Note    --The editor has 2 modes--
 
             1) edit  mode (default mode)
             2) input mode
 
        --When the editor is in    edit mode,
          then all of the editor commands are
          at the disposal of the analyst.
          The     edit mode    is the most common
          mode of operation; most changing and updating
          of files is done in    edit mode.
          The advantage of    edit mode   is that
          all of the editor commands may be used to
          make the desired changes in the file.
          The disadvantage of    edit mode is that
          extra keystrokes are required for the entry
          of large amounts of text.
 
        --When the editor is in    input mode,
          then "what you type" is "what goes into
          the file".  Input mode   is the most
          common way of entering large amounts
          of text into a file.
          The advantage of    input mode   is that
          large amounts of text may be efficiently
          entered.   The disadvantage of    input mode
          is that the analyst has no immediate access
          to editor commands (such as PRINT, DELETE,
          and so forth).
 
        --To go from edit mode to input mode, enter
 
             INPUT
 
        --To go from input mode to edit mode, enter
 
             EDIT
 
        --When in    input mode, the "current line"
          seting is updated as text lines are entered.
 
Synonyms--None
 
Examples--INPUT
 
Related Commands--
          EDIT            = Goes from input mode to edit mode.
 
----------------------------------------------------------
 
INSERT
 
Command --INSERT
 
Purpose --Inserts a string immediately after
          the current line.
 
Format  --INSERT   [string]
 
          Inserts [string] immediately after the
          current line.
 
Note    --The INSERT command is a heavily used command;
          the full INSERT command spelling is, however,
          rarely used; the    I    synonym is much more
          common.  Thus to insert the string
             The quick brown fox
          immediately after the current line,
          one could enter--
             I The quick brown fox
 
        --The INSERT command provides an
          alternative way of entering text
          (aside from switching into    input mode
          via the INPUT command).  If only a few
          lines of text need be entered, then use
          of the INSERT command is efficient because
          the text is getting inserted and yet
          the analyst never leaves   edit mode   and
          so has the advantage of having at disposal
          all of the editor commands.  On the other
          hand, if much text needs to be entered, then
          it makes sense to enter    input mode
          and type away continuously until all of
          the text has been entered.  The    input mode
          method of inserting the above 1 line is--
             INPUT
             The quick brown fox
             EDIT
 
        --The INSERT command is sensitive to
          the auto-tab setting (as set via the
          SET TAB command).  Thus if it is desired
          that all text start in column 7, then
          one method of achieving that is to enter
             SET TAB 7
          before entering various test lines
          (via INSERT or via input mode).
          The resulting inserted lines
          will have spaces automatically
          inserted so that the first
          character of each line will start
          at column 7.  The default
          SET TAB setting is column 1.
 
        --Another way to achieve a similar effect
          is to leave the auto-tab setting at 1,
          enter (via INSERT or via input mode) the
          desired text lines, and then (after
          all the text is entered), shift the
          text to the desired column via the
          SET INDENT and INDENT commands, as in
             SET INDENT 7
             INDENT 20 30
          which would set the indentation column
          to 7, and then indent lines 20 to 30,
          say, of the text.
 
        --The INSERT command with an empty string
          provides a convenient way of entering
          a blank line.
 
        --Immediately after the INSERT command
          has been executed, the "current line"
          setting is updated by 1, so that the
          analyst is thus "pointing at" the
          line which just got inserted.
 
Synonyms--I
          IN
 
Examples--I ABC               Inserts ABC
          I 100 CONTINUE      Inserts 100 CONTINUE
          I                   Inserts a blank line.
 
Related Commands--
          SET TAB         = Sets auto-tab column.
          SET INDENT      = Sets indentation column.
          SET CENTER      = Sets centering column.
          SET TRUNCATE    - Set truncation column.
          INDENT          = Indents lines of text.
          CENTER          = Centers lines of text.
          TRUNCATE        = Truncates lines of text.
          DELETE          = Deletes k lines (start with current).
          PRINT           = Prints k lines (start with current).
          PA              = Prints all lines (start with current).
          SET FEEDBACK    = Allows/suppresses printing.
          SET NUMBER      = Allows/suppresses line numbers.
          SET PROMPT      = Allows/suppr. prompt & line num.
 
----------------------------------------------------------
 
 
 
 
 
 
 
 
 
 
 
 
 
 
 
 
 
 
 
 
 
 
 
 
 
 
 
 
 
 
 
 
 
 
 
 
 
 
 
 
 
 
 
 
 
 
 
 
 
----------------------------------------------------------
-------------------------  *J*  --------------------------
----------------------------------------------------------
 
 
 
 
 
 
 
 
 
 
 
 
 
 
 
 
 
 
 
 
 
 
 
 
 
 
 
 
 
 
 
 
 
 
 
 
 
 
 
 
 
 
 
 
 
 
 
 
 
 
 
 
 
 
 
 
 
 
 
 
 
 
 
 
 
 
 
 
 
 
 
 
 
 
 
 
 
 
 
 
 
 
 
 
 
 
 
 
 
 
 
 
 
 
 
 
 
----------------------------------------------------------
-------------------------  *K*  --------------------------
----------------------------------------------------------
 
 
 
 
 
 
 
 
 
 
 
 
 
 
 
 
 
 
 
 
 
 
 
 
 
 
 
 
 
 
 
 
 
 
 
 
 
 
 
 
 
 
 
 
 
 
 
 
 
 
 
 
 
 
 
 
 
 
 
 
 
 
 
 
 
 
 
 
 
 
 
 
 
 
 
 
 
 
 
 
 
 
 
 
 
 
 
 
 
 
 
 
 
 
 
 
 
----------------------------------------------------------
-------------------------  *L*  --------------------------
----------------------------------------------------------
 
LAST
 
Command --LAST
 
Purpose --Goes to the last line of the file
          and prints it out (on the screen).
 
Format  --LAST
 
Note    --The "current line" setting is updated
          to the     number of lines     in the file.
 
        --The LAST command is a convenent way to
          determine the number of lines currently
          in the file.
 
Synonyms--LA
 
Examples--LAST
          LAS
 
Related Commands--
          TOP             = Goes before line 1.
          BOTTOM          = Goes after last line.
          FIRST           = Goes to first line.
          GO              = Goes to any line.
          #               = Goes to any line.
          SET NUMBER      = Prints line numbers.
          SET PROMPT      = Prints prompt & line number.
 
----------------------------------------------------------
 
LOCATE
 
Command --LOCATE
 
Purpose --Locates the next line on which
          the specified string occurs, and
          prints the line out (on the screen).
 
Format  --The LOCATE command has 2 forms--
 
       1--LOCATE
 
          This no-argument form locates the
          next line on which the previously-
          specified string occurs and
          prints the line out (on the screen).
          The "previously-specified string" is
          the string that was used in the
          previous LOCATE command.  Thus the
          no-argument form of the LOCATE command
          "remembers" previous located strings
          and uses them if the analyst so desires.
 
       2--LOCATE     [string]
 
          This 1-argument form locates the
          next line on which the specified
          string occurs and prints the line
          out (on the screen).
 
Note    --The search for the string
          does not include the current line;
          the search starts on the next line.
          If the string is found anywhere on
          some line, the search is terminated,
          and the line is printed out.
 
        --Each line is searched its full width
          (the default column limits for the search
          are columns 1 to 132).
          To restrict the search to a certain range  of
          columns (e.g., columns 7 through 72), then
          use the SET command and its LLIMITS sub-command
          before using the LOCATE command, as in
             SET LLIMITS 7 72
             LOCATE ABC
 
        --The LOCATE command finds strings anywhere
          in the pre-specified columns limits.  If it
          is important to restrict the search so
          that a "hit" is obtained only if the string
          starts in a specific column (e.g., an A in column
          11, a B in column 12, and a C in column 13,
          then use the FIND command, as in
             FIND           ABC
          The spacing in the FIND command is significant.
 
        --For both forms of the LOCATE command, after the
          command has been executed, the "current line"
          setting is updated to the line that was
          printed.
 
        --The LOCATE command is a heavily used command;
          the full LOCATE command spelling is, however,
          rarely used; the    L    synonym is much more
          common.  Thus to find the next occurrance of
          the string     1500 CONTINUE
          it is rare to use
             LOCATE 1500 CONTINUE
          it is more common to use
             L 1500 CONTINUE
 
Synonyms--L
 
Examples--LOCATE ABC       Locate ABC
          L ABC            Locate ABC
 
Related Commands--
          SET LLIMITS     = Sets LOCATE column limits.
          FIND            = Finds a string starting in a specific column.
          CHANGE          = Changes a string to another string.
          PRINT           = Prints k lines (start with current).
          SET FEEDBACK    = Allows/suppresses printing.
          SET NUMBER      = Allows/suppresses line numbers.
          SET PROMPT      = Allows/suppr. prompt & line num.
 
----------------------------------------------------------
 
LI
 
Command --LI (Locate and Insert)
 
Purpose --Locates all subsequent lines
          which contain a specified string, and
          inserts immediately after such
          lines a new line consisting of XXX.
 
Format  --The LI command has 2 forms--
 
       1--LI     [old string]
 
          This 1-argument form locates
          all subsequent lines which
          contain [old string] and inserts
          immediately after such lines
          a new line consisting of XXX.
 
       2--LI     [old string]     [new string]   <<<not operational>>>
 
          This 2-argument form locates
          all subsequent lines which
          contain [old string] and inserts
          immediately after such lines
          a new line consisting of [new string].
 
Note    --The search for the string
          does not start until the next line.
          If the string is found anywhere on
          such subsequent lines, a new
          line is inserted, and the
          search resumes.
 
        --Each line is searched its full width
          (the default column limits for the search
          are columns 1 to 132).
          To restrict the search to a certain range  of
          columns (e.g., columns 7 through 72), then
          use the SET LOCATE command before using
          the LI command, as in
             SET LOCATE 7 72
             LI ABC DEF
          which will search columns 7 through 72
          of all subsequent lines, and anytime
          an ABC is found on such lines, a new
          line consisting of
             DEF
          is inserted immediately after each
          such line.
 
        --[old string] and [new string] in the
          above forms must each consist of
          contiguous characters; that is, there
          may be at most 1 or 2 arguments--
          never more; thus
             LI FED XXXX
          is legal (there are only 2 disinct
          words--FED and XXXX), but
             LI FED CALL FIX.ED
          is illegal (because there are 3 distinct
          words--FED, CALL, and FIX.ED).
 
        --In practice, the restriction that
          [new string] be contiguous characters is
          of very little consequence.
          If a multi-word new string is desired,
          then it is common in practice
          to use LI command with [new string]
          being a unique 1-word entry
          (such as XXXX), and then followsing up
          with a CHANGE command to change
          all XXXX's to the desired multi-word
          result, as in
             LI ABC XXXX
             TOP
             CHANGE /XXXX/CALL FIX.ED/  55555
          which would insert the line XXXX
          after each line containing ABC,
          then go back to the top of the file,
          and then change XXXX on any line
          to CALL FIX.ED (for the next 55555 lines--
          which in practice means the whole file).
 
        --As the above example alludes to,
          the LI command is a convenient way
          (starting with a list of all files
          in a directory) of setting up a
          system runstream to systematically
          re-edit all subroutines in a
          directory.  In the above example,
          the re-editing is specified by EDIT/FED
          commands residing in the file FIX.ED   .
          With very little difficulty, one could
          end up with a file consisting of (for a VAX)
             $FED SUB1.FOR
             CALL FIX.ED
             $FED SUB2.FOR
             CALL FIX.ED
             $FED SUB3.FOR
             CALL FIX.ED
             etc.
          which one could then execute (on a VAX)
          by using the VAX @ utility.
 
        --For both forms of the LI command, after the
          command has been executed, the "current line"
          setting is updated to the bottom of the file.
 
Synonyms--None.
 
Examples--LI ABC               Locate every subsequent
                               line containing ABC and
                               insert a line consisting
                               of XXX
        --LI CONTINUE C        Locate every subsequent
                               line containing CONTINUE and
                               insert a line consisting
                               of C
 
Related Commands--
          LC              = Locate and Call.
          LOCATE          = Locates a string.
          SET LOCATE      = Sets column limits for
                            LOCATE, PTL, DTL, and
                            LI commands.
          PTL             = Print-till-locate =
                            print from current line down
                            to line with located string.
          DTL             = Delete-till-locate =
                            delete from current line down
                            to line with located string.
          CHANGE          = Changes strings on lines.
          PRINT           = Prints k lines (start with current).
          SET FEEDBACK    = Allows/suppresses printing.
          SET NUMBER      = Allows/suppresses line numbers.
          SET PROMPT      = Allows/suppr. prompt & line num.
 
----------------------------------------------------------
 
LC
 
Command --LC (Locate and Call)
 
Purpose --Locates all subsequent lines
          which contain a specified string, and
          calls the specified macro
          whenever the string is found.
 
Format  --LC     [string]   [macro file]
 
          This form locates
          all subsequent lines which
          contain [string] and calls
          [macro file] each time the
          string is found.
 
Note    --The search for the string
          does not start until the next line.
          If the string is found anywhere on
          such subsequent lines, the macro
          is called (executed), and the
          search resumes.
 
        --Each line is searched its full width
          (the default column limits for the search
          are columns 1 to 132).
          To restrict the search to a certain range  of
          columns (e.g., columns 7 through 72), then
          use the SET LOCATE command before using
          the LI command, as in
             SET LOCATE 7 72
             LC ABC DEF
          which will search columns 7 through 72
          of all subsequent lines, and anytime
          an ABC is found on such lines, a new
          line consisting of
             DEF
          is inserted immediately after each
          such line.
 
        --[string] in the
          above form must consist of
          contiguous characters; that is, there
          must be exactly a total of 2 arguments--
          thus
             LC FED FIX1.MAC
          is legal (there are only 2 disinct
          words--FED and FIX1.MAC), but
             LC FED CALL FIX1.MAC
          is illegal (because there are 3 distinct
          words--FED, CALL, and FIX1.MAC).
 
        --As the above example alludes to,
          the LC command is a convenient way
          (starting with a list of all files
          in a directory) of setting up a
          system runstream to systematically
          re-edit all subroutines in a
          directory.  In the above example,
          the re-editing is specified by EDIT/FED
          commands residing in the file FIX.MAC   .
          With very little difficulty, one could
          end up with a file consisting of (for a VAX)
             $FED SUB1.FOR
             CALL FIX.MAC
             $FED SUB2.FOR
             CALL FIX.MAC
             $FED SUB3.FOR
             CALL FIX.MAC
             etc.
          which one could then execute (on a VAX)
          by using the VAX @ utility.
 
        --After the LC command is executed, the
          "current line" setting is updated
          depending on what the operations
          that were specified inside the macro.
 
Synonyms--None.
 
Examples--LC ABC FIX.MAC       Locate every subsequent
                               line containing ABC and
                               (each time the string is found)
                               call the macro residing
                               in the file   FIX.MAC.
 
Related Commands--
          LI              = Locate and Insert.
          LOCATE          = Locates a string.
          SET LOCATE      = Sets column limits for
                            LOCATE, PTL, DTL, and
                            LI commands.
          PTL             = Print-till-locate =
                            print from current line down
                            to line with located string.
          DTL             = Delete-till-locate =
                            delete from current line down
                            to line with located string.
          CHANGE          = Changes strings on lines.
          PRINT           = Prints k lines (start with current).
          SET FEEDBACK    = Allows/suppresses printing.
          SET NUMBER      = Allows/suppresses line numbers.
          SET PROMPT      = Allows/suppr. prompt & line num.
 
----------------------------------------------------------
 
LIST
 
Command --LIST
 
Purpose --Lists (onto the screen) the
          contents of an external file.
 
Format  --LIST   [file name]
 
          This lists the contents of the file
          with the name   [file name].  The LIST
          command thus allows the analyst to carry
          out an editing session in one file and
          (without leaving the editor) peruse the
          contents of another file.
 
Note    --After the LIST command has been
          executed, the "current line" setting
          in the primary file remains unchanged.
 
Synonyms--None.
 
Examples--LIST FIX.ED         Print the contents
                              of the file FIX.ED
        --LIST REPORT.TEX     Print the contents
                              of the file REPORT.TEX
 
Related Commands--
          ADD             = Adds (= copy into the current file)
                            the contents of a second file.
          CALL            = Calls (= executes) the macro
                            residing in a second file.
          COPY            = Copies out the contents of all (or part)
                            of current file into another file.
          PRINT           = Prints k lines (of current file).
          PA              = Prints all lines of current file.
          SET PRINTER     = Activates hardcopy Printer.
          SET NUMBER      = Turns line numbers on/off.
          SET PROMPT      = Turns prompt on/off.
 
----------------------------------------------------------
 
 
 
 
 
 
 
 
 
 
 
 
 
 
 
 
 
 
 
 
 
 
 
 
 
 
 
 
 
 
 
 
 
 
 
 
 
 
 
 
 
 
 
 
 
 
 
 
 
 
 
 
 
 
 
 
 
 
 
 
 
 
 
 
 
 
 
 
 
 
 
 
 
 
 
 
 
 
 
 
 
 
 
 
 
 
 
 
----------------------------------------------------------
-------------------------  *M*  --------------------------
----------------------------------------------------------
 
MOVE
 
Command --MOVE
 
Purpose --Moves (= copies and then deletes) a block of text
             1) to the editor's temporary copy file; or
             2) to an external file; or
             3) to another location in the file.
 
Format  --The MOVE command has 3 forms--
 
       1--MOVE
          This no-argument form moves the
          previously-specified block of text
          to the editor's temporary copy file.
          The previous-specification is done via the
          SET BEGIN, SET END, or SET COPY  commands).
          The "current line" setting is updated to
          be the first line of the text block that
          was moved.
 
       2--MOVE    [file name]
          This 1-argument form moves the
          previously-specified block of text
          to the specified file.
          The previous-specification is done via the
          SET BEGIN, SET END, or SET COPY  commands).
          The "current line" setting is updated to
          be the first line of the text block that
          was moved.
 
       3--MOVE   [start line]   [stop line]   [target line]
          This 3-argument form moves the block of text
          from the    start line   through the    stop line
          and inserts this block immediately after
          the    target line.
          The "current line" setting is updated to
          be the last line of the moved block in its
          new location
 
Note    --The MOVE command is like a combination of
          the COPY and DELETE commands--the original
          version of the text block is deleted after
          it is moved.
 
Note    --If the number of lines moved is
          less than 10, then all lines
          will be printed on the screen
          as they are being moved.
        --If the number of lines moved
          is greater than 10, then only the
          first and last moved lines will
          be printed.
        --If no printing is desired, then
          the FEEDBACK OFF command should
          be entered beforehand.
 
Synonyms--MO
          DCOPY
          DCO
 
Examples--3 ways to move the text from
          lines 50 to 60 and insert it
          immediately after line 20--
 
          1) SET COPY   50 60
             MOVE
             20
             ADD
 
          2) SET COPY   50 60
             MOVE YOURFILE.TEX
             20
             ADD YOURFILE.TEX
 
          3) MOVE  50 60 20
 
Related Commands--
          SET BEGIN       = Defines first line of block.
          SET END         = Defines last line of block.
          SET COPY        = Defines first & last lines of block.
          ADD             = Adds (= insert) a block of text.
          COPY            = Copy a block of text.
          SET FEEDBACK    = Allows/suppresses printing.
 
----------------------------------------------------------
 
 
 
 
 
 
 
 
 
 
----------------------------------------------------------
-------------------------  *N*  --------------------------
----------------------------------------------------------
 
NEXT
 
Command --NEXT
 
Purpose --Goes to the next line (or the next
          k-th line) of the file, and
          prints the line out (on the screen).
 
Format  --The NEXT command has 2 forms--
 
       1--NEXT
 
          This no-argument form goes to the
          next line of the file and prints it
          (on the screen).
 
       2--NEXT     [number of lines]
 
          This 1-argument form goes down
          [number of lines] lines in the file and
          prints (on the screen) that line.
 
Note    --For both forms, after the NEXT command
          has been executed, the "current line"
          setting is updated to the line that
          was printed.
 
        --The NEXT command is a heavily used command;
          the NEXT command spelling is, however, rarely used;
          N    and    carriage return    are more common.
 
        --To go to the next line, it is rare to use
             NEXT
          it is more common to use
             N
          or more simply just
             carriage return
 
        --To go down, say, 7 lines in a file,
          It is rare to use
             NEXT 7
          it is more common to use
             N 7
 
Synonyms--N
          carriage return
 
Examples--NEXT               Go down 1 line.
          N                  Go down 1 line.
          carriage return    Go down 1 line.
          NEXT 12            Go down 12 lines.
          N 12               Go down 12 lines.
 
Related Commands--
          UP              = Goes up k lines.
          TOP             = Goes before line 1.
          FIRST           = Goes to first line.
          LAST            = Goes to last line.
          BOTTOM          = Goes after last line.
          GO              = Goes to any line.
          #               = Goes to any line.
          PRINT           = Prints k lines (start with current).
          PRINT-NEXT (PN) = Prints k lines (start with next).
          SET NUMBER      = Allows/suppresses line numbers.
          SET PROMPT      = Allows/suppr. prompt & line num.
 
----------------------------------------------------------
 
 
 
 
 
 
 
 
 
 
 
 
 
 
 
 
 
 
 
 
 
 
 
 
 
 
 
 
 
----------------------------------------------------------
-------------------------  *O*  --------------------------
----------------------------------------------------------
 
 
 
 
 
 
 
 
 
 
 
 
 
 
 
 
 
 
 
 
 
 
 
 
 
 
 
 
 
 
 
 
 
 
 
 
 
 
 
 
 
 
 
 
 
 
 
 
 
 
 
 
 
 
 
 
 
 
 
 
 
 
 
 
 
 
 
 
 
 
 
 
 
 
 
 
 
 
 
 
 
 
 
 
 
 
 
 
 
 
 
 
 
 
 
 
 
----------------------------------------------------------
-------------------------  *P*  --------------------------
----------------------------------------------------------
 
PRINT
 
Command --PRINT
 
Purpose --Prints specified lines onto the screen.
 
Format  --The PRINT command has 3 forms--
 
       1--PRINT
 
          This no-argument form prints the
          current line only.
 
       2--PRINT [number of lines]
 
          The 1 argument form prints the specified
          [number of lines].  The printing starts with
          the current line.
 
       3--PRINT    [start line]    [stop line]
 
          The 2-argument form prints the specified
          range of lines, that is prints from the
          [start line]     through the     [stop line].
 
Note    --For all 3 forms, after the PRINT command
          has been executed, the "current line" is
          updated to be the last line printed.
 
Synonyms--P
 
Examples--P        Print current line.
          P 10     Print next 10 lines.
                   (starting with current line).
          P 30 50  Print lines 30 to 50.
 
Related Commands--
          PTL             = Print-till-locate =
                            prints from current line down
                            to line with located string.
          PP              = Print a pageful.
          PA              = Prints all remaining lines
                            (to the end of text).
          PN              = Print lines (but start
                            with next line).
          SET PRINTER     = Activates hardcopy Printer.
          SET NUMBER      = Turns line numbers on/off.
          SET PROMPT      = Turns prompt on/off.
          NEXT            = Goes to (and Print) next line.
 
----------------------------------------------------------
 
PA
 
Command --PA
 
Purpose --Prints specified line and all
          remaining lines onto the screen.
 
Format  --The PA command has 2 forms--
 
       1--PA
 
          This no-argument form prints the
          current line and all remaining lines.
 
       2--PA [start line]
 
          The 1 argument form prints the [start line]
          and all remaining lines.
 
Note    --For all 2 forms, after the PA command
          has been executed, the "current line" is
          updated to be the last line printed.
 
Synonyms--None
 
Examples--PA       Print current line to end of file.
          PA 10    Print line 10 to end of file.
 
Related Commands--
          PRINT           = Prints specified lines.
          PTL             = Print-till-locate =
                            prints from current line down
                            to line with located string.
          PP              = Print a pageful.
          PN              = Print lines (but start
                            with next line).
          SET PRINTER     = Activates hardcopy Printer.
          SET NUMBER      = Turns line numbers on/off.
          SET PROMPT      = Turns prompt on/off.
          NEXT            = Goes to (and Print) next line.
 
----------------------------------------------------------
 
PTL
 
Command --PTL (Print till Locate)
 
Purpose --Locates the next line on which
          the specified string occurs, and
          prints all lines from the current line
          down to the line with the located string.
 
Format  --The PTL command has 2 forms--
 
       1--PTL
 
          This no-argument form locates the
          next line on which the previously-
          specified string occurs, and
          prints all lines from the current line
          down to the line with the located string.
          The "previously-specified string" is
          the string that was explicitly used in a
          previous LOCATE, PTL, DTL, or LAI command.
          Thus the no-argument form of the PTL command
          "remembers" previous located strings
          and uses them if the analyst so desires.
 
       2--PTL     [string]
 
          This 1-argument form locates the
          next line on which the specified
          string occurs and  prints all lines
          from the current line down to the line
          with the located string.
 
Note    --The printing starts on the current line;
          but the search for the string
          does not start until the next line.
          If the string is found anywhere on
          some line, the search is terminated,
          and the lines are printed out.
 
        --Each line is searched its full width
          (the default column limits for the search
          are columns 1 to 132).
          To restrict the search to a certain range  of
          columns (e.g., columns 7 through 72), then
          use the SET LOCATE command before using
          the PTL command, as in
             SET LOCATE 7 72
             PTL ABC
          which will print the current line
          and every line down to the first
          line containing ABC in columns
          7 through 72.
 
        --If the specified string is not located,
          then the current line and all subsequent
          lines are printed.
 
        --For both forms of the PTL command, after the
          command has been executed, the "current line"
          setting is updated to the last line that was
          printed.
 
Synonyms--None.
 
Examples--PTL ABC              Print current line down
                               to the first line containing
                               the string     ABC
        --PTL 1000 CONTINUE    Print current line down
                               to the first line containing
                               the string     1000 CONTINUE
 
Related Commands--
          SET LOCATE      = Sets column limits for
                            LOCATE, PTL, DTL, and
                            LAI commands.
          LOCATE          = Locates a string.
          DTL             = Delete-till-locate =
                            delete from current line down
                            to line with located string.
          LAI             = Locate-all-and-insert =
                            locate all lines with a string
                            and insert new lines with a second
                            string.
          PRINT           = Prints k lines (start with current).
          SET FEEDBACK    = Allows/suppresses printing.
          SET NUMBER      = Allows/suppresses line numbers.
          SET PROMPT      = Allows/suppr. prompt & line num.
 
----------------------------------------------------------
 
 
 
 
 
 
 
 
 
 
 
----------------------------------------------------------
-------------------------  *Q*  --------------------------
----------------------------------------------------------
 
 
 
 
 
 
 
 
 
 
 
 
 
 
 
 
 
 
 
 
 
 
 
 
 
 
 
 
 
 
 
 
 
 
 
 
 
 
 
 
 
 
 
 
 
 
 
 
 
 
 
 
 
 
 
 
 
 
 
 
 
 
 
 
 
 
 
 
 
 
 
 
 
 
 
 
 
 
 
 
 
 
 
 
 
 
 
 
 
 
 
 
 
 
 
 
 
----------------------------------------------------------
-------------------------  *R*  --------------------------
----------------------------------------------------------
 
 
 
 
 
 
 
 
 
 
 
 
 
 
 
 
 
 
 
 
 
 
 
 
 
 
 
 
 
 
 
 
 
 
 
 
 
 
 
 
 
 
 
 
 
 
 
 
 
 
 
 
 
 
 
 
 
 
 
 
 
 
 
 
 
 
 
 
 
 
 
 
 
 
 
 
 
 
 
 
 
 
 
 
 
 
 
 
 
 
 
 
 
 
 
 
 
----------------------------------------------------------
-------------------------  *S*  --------------------------
----------------------------------------------------------
 
SHIFT
 
Command --SHIFT
 
Purpose --Shifts existing text in the specified range
          of lines so that each line is displaced
          (left or right) by the specified target
          amount.
 
Format  --The SHIFT command has 3 forms--
 
       1--SHIFT
 
          This no-argument form shifts the current
          line (only) so that the resulting text
          is displaced (left or right) by
          the specified target amount.
 
       2--SHIFT     [number of lines]
 
          This 1-argument form shifts each line
          for the specified number of lines.  The
          shifting starts with the current line.
          Each line is displaced (left or right) by
          specified target amount.
 
       3--SHIFT   [first line]   [last line]
 
          This 2-argument form shifts each line
          for the specified range of lines; that is,
          shifting starts with the   [first line]
          and proceeds through the   [last line].
          Each line is displaced (left or right) by
          specified target amount.
 
Note    --For all 3 cases, the    [target amount]
          is pre-specified by prior use of the
          SET SHIFT command, as in
             SET SHIFT 10
             SHIFT 100 120
          which would set the [target amount] to 10
          columns to the right and then would
          shift lines 100 through 120.
 
        --Negative values in the SET SHIFT command
          indicate a displacement to the left; thus
             SET SHIFT -2
             SHIFT 100 120
          would set the [target amount] to -2
          (that is, 2 columns to the left) and then
          would shift lines 100 through 120.
 
        --The shifting operation takes the existing
          text string on each line, and displaces (left
          or right) the entire line by the specified
          [target amount].
 
        --SHIFT is similar to INDENT in that
          both result in text lines being shifted.
          SHIFT differs from INDENT in that
          SHIFT merely does a straightforward
          translation (left or right), wheras
          INDENT assures that the resulting text
          will start in a specified column,
 
        --The SHIFT command is useful for the
          shifting of lists and the like.
 
        --The SHIFT command operates on existing text,
          text which is already in the specified line.
          It has no effect on yet-to-be-entered text.
 
        --For all 3 forms, after the SHIFT command
          has been executed, the "current line" is
          updated to be the last line shifted.
 
Synonyms--SHI
 
Examples--SET SHIFT 6       Specifies shift amount as
                            6 columns to the right.
          SHIFT             Shift current line right 6 columns.
          SHIFT 10          Shift next 10 lines right 6 columns
                            (starting with current line).
          SHIFT 10 15       Shift lines 10 to 15 right 6 columns.
 
Related Commands--
          SET SHIFT       = Specifies shift amount.
          SET INDENT      = Specifies indentation column.
          INDENT          = Indents text on lines.
          CENTER          = Centers text on lines.
          TRUNCATE        = Truncates text on lines.
          SET TAB         = Specifies auto-tabbing column.
          PRINT           = Prints lines.
          SET FEEDBACK    = Allows/suppresses screen printing.
 
 
----------------------------------------------------------
 
SPLIT
 
Command --SPLIT
 
Purpose --Deletes all characters on the current
          line beyond (and including) the
          specified string, and places the
          deleted string at the beginning
          of the text on the next line.
 
Format  --SPLIT   [string]
 
          Deletes all characters on the current
          line beyond (and including) [string],
          and places the deleted string at
          the beginning of the text on
          the next line.
 
Note    --If multiple occurrances of the string
          are on the line, then the splitting will
          be done from the last (= right-most) such
          occurrance.  Thus if the original line is
             ABC DEF GHI ABC DEF GHI
          then the command
             SPLIT DEF
          will result in
             ABC DEF GHI ABC
          The commands
             SPLIT DE
          and
             SPLIT D
          will yield the same result.
 
        --If the following line already has
          text on it, then the split part
          from the previous line will be
          inserted at the beginning of the
          next line at the first non-blank
          character.  Also, a space will
          automatically be inserted between
          the end of the inserted split
          and the beginning of the text already
          on the line (it is usually desired to
          have such an intervening space).  In
          the previous example, the split part was
             DEF GHI
          If the next line is
                AAAAA BBBBB CCCCC
          then the new next line will be
                DEF GHI AAAAA BBBBB CCCCC
          If the next line is
                         AAAAA BBBBB CCCCC
          then the new next line will be
                         DEF GHI AAAAA BBBBB CCCCC
 
        --If the following line is blank, then
          the blank line will be left as is,
          and a new line will be created between
          the first line and the blank line.
          On this newly-created line, the
          insertion will obey current
          auto-tab setting (as set via
          prior use of the SET TAB command).
          The default auto-tab setting of 1
          would lead to the split part
          being inserted at column 1.  If the
          analyst had previously entered
             SET TAB 11
          then the split part would have started
          in column 11.
 
        --After the SPLIT command has
          been executed, the "current line"
          setting will be 1 larger than
          the original current line setting.
 
Synonyms--SP
 
Examples--SPLIT ABC          Splits ABC and beyond
                             on current line and
                             places ABC on the next line.
        --SPLIT ing          Splits   ing    and beyond
                             on current line and
                             places    ing    on the next line.
 
Related Commands--
          EXTEND          = Append string to end of line.
          SPLIT           = Splits string onto next line.
          SET TAB         = Sets auto-tab column.
          SET INDENT      = Sets indentation column.
          SET CENTER      = Sets centering column.
          SET TRUNCATE    = Set truncation column.
          SET SHIFT       = Set shift column.
          INDENT          = Indents lines of text.
          CENTER          = Centers lines of text.
          TRUNCATE        = Truncates lines of text.
          SHIFT           = Shifts lines of text.
          DELETE          = Deletes k lines (start with current).
          PRINT           = Prints k lines (start with current).
          SET FEEDBACK    = Allows/suppresses printing.
          SET NUMBER      = Allows/suppresses line numbers.
          SET PROMPT      = Allows/suppr. prompt & line num.
 
----------------------------------------------------------
 
 
 
 
 
 
 
 
 
 
 
 
 
 
 
 
 
 
 
 
 
 
 
 
 
 
 
 
 
 
 
 
 
 
 
 
 
 
 
 
 
 
 
 
 
 
 
 
 
 
 
 
 
 
 
 
 
 
 
 
 
 
 
 
 
 
 
 
 
 
 
 
 
 
 
 
 
 
 
 
 
 
 
 
 
 
 
 
 
 
 
 
 
 
----------------------------------------------------------
-------------------------  *T*  --------------------------
----------------------------------------------------------
 
TOP
 
Command --TOP
 
Purpose --Goes to the (imaginary) line immediately
          before the first line of the file.
 
Format  --TOP
 
Note    --The "current line" setting is changed
          to the value 0.
 
        --[TOP]     appears on the screen.
 
Synonyms--T
 
Examples--T
          TOP
 
Related Commands--
          FIRST           = Goes to first line.
          LAST            = Goes to last line.
          BOTTOM          = Goes after last line.
          GO              = Goes to any line.
          #               = Goes to any line.
          SET NUMBER      = Prints line numbers.
          SET PROMPT      = Prints prompt & line number.
 
----------------------------------------------------------
 
TRUNCATE
 
Command --TRUNCATE
 
Purpose --Truncates existing text in the specified range
          of lines so that each line is truncated to
          the specified target column.
 
Format  --The TRUNCATE command has 3 forms--
 
       1--TRUNCATE
 
          This no-argument form truncates the current
          line (only) so that the resulting text
          is truncated to the target column.
 
       2--TRUNCATE     [number of lines]
 
          This 1-argument form truncates each line
          for the specified number of lines.  The
          truncating starts with the current line.
          Each line is truncated to the target column.
 
       3--TRUNCATE   [first line]   [last line]
 
          This 2-argument form truncates each line
          for the specified range of lines; that is,
          truncating starts with the   [first line]
          and proceeds through the   [last line].
          Each line is truncated to the target column.
 
Note    --For all 3 cases, the    [target column]
          is pre-specified by prior use of the
          SET TRUNCATE command, as in
             SET TRUNCATE 80
             TRUNCATE 100 120
          which would set the [target column] to 80
          and then truncate lines 100 through 120.
 
        --The truncating operation takes the existing
          text sting on each line, determines the last
          non-blank character, and deletes (if necessary)
          all characters in the text string beyond
          the     [target column].
 
        --The TRUNCATE command is useful for removing
          "garbage" text out at the end of lines.
 
        --The TRUNCATE command operates on existing text,
          text which is already in the specified line.
          It has no effect on yet-to-be-entered text.
 
        --For all 3 forms, after the TRUNCATE command
          has been executed, the "current line" is
          updated to be the last line truncated.
 
Synonyms--TRUN
 
Examples--SET TRUNCATE 72    Specifies truncation column as 72.
          TRUNCATE           Truncate current line to column 72.
          TRUNCATE 10        Truncate next 10 lines to column 72.
                             (starting with current line).
          TRUNCATE 10 15     Truncate lines 10 to 15 to column 72.
 
Related Commands--
          SET TRUNCATE    = Specifies the truncation column.
          INDENT          = Indents text on lines.
          CENTER          = Centers text on lines.
          SHIFT           = Shifts text on lines.
          SET TAB         = Specifies column for auto-tabbing.
          PRINT           = Prints lines.
          SET FEEDBACK    = Allows/suppresses screen printing.
 
----------------------------------------------------------
 
 
 
 
 
 
 
 
 
 
 
 
 
 
 
 
 
 
 
 
 
 
 
 
 
 
 
 
 
 
 
 
 
 
 
 
 
 
 
 
 
 
 
 
 
 
 
 
 
 
 
 
 
 
 
 
 
 
 
 
 
 
 
 
 
 
 
 
 
 
 
 
 
 
 
 
 
 
 
 
 
 
 
 
 
 
 
 
 
 
 
 
----------------------------------------------------------
-------------------------  *U*  --------------------------
----------------------------------------------------------
 
UP
 
Command --UP
 
Purpose --Goes to the previous line (or the
          k-th previous line) of the file, and
          prints the line out (on the screen).
 
Format  --The UP command has 2 forms--
 
       1--UP
 
          This no-argument form goes to the
          previous line of the file and prints it
          (on the screen).
 
       2--UP     [number of lines]
 
          This 1-argument form goes up
          [number of lines] lines in the file and
          prints (on the screen) that line.
 
Note    --For both forms, after the UP command
          has been executed, the "current line"
          setting is updated to the line that
          was printed.
 
        --The UP command is a heavily used command;
 
Synonyms--U
 
Examples--UP        Go up 1 line.
          U         Go up 1 line.
          UP 12     Go up 12 lines.
          U 12      Go up 12 lines.
 
Related Commands--
          NEXT            = Goes down k lines.
          TOP             = Goes before line 1.
          FIRST           = Goes to first line.
          LAST            = Goes to last line.
          BOTTOM          = Goes after last line.
          GO              = Goes to any line.
          #               = Goes to any line.
          PRINT           = Prints k lines (start with current).
          PRINT-NEXT (PN) = Prints k lines (start with next).
          SET NUMBER      = Allows/suppresses line numbers.
          SET PROMPT      = Allows/suppr. prompt & line num.
 
----------------------------------------------------------
 
 
 
 
 
 
 
 
 
 
 
 
 
 
 
 
 
 
 
 
 
 
 
 
 
 
 
 
 
 
 
 
 
 
 
 
 
 
 
 
 
 
 
 
 
 
----------------------------------------------------------
-------------------------  *V*  --------------------------
----------------------------------------------------------
 
 
 
 
 
 
 
 
 
 
 
 
 
 
 
 
 
 
 
 
 
 
 
 
 
 
 
 
 
 
 
 
 
 
 
 
 
 
 
 
 
 
 
 
 
 
 
 
 
 
 
 
 
 
 
 
 
 
 
 
 
 
 
 
 
 
 
 
 
 
 
 
 
 
 
 
 
 
 
 
 
 
 
 
 
 
 
 
 
 
 
 
 
 
 
 
 
----------------------------------------------------------
-------------------------  *W*  --------------------------
----------------------------------------------------------
 
 
 
 
 
 
 
 
 
 
 
 
 
 
 
 
 
 
 
 
 
 
 
 
 
 
 
 
 
 
 
 
 
 
 
 
 
 
 
 
 
 
 
 
 
 
 
 
 
 
 
 
 
 
 
 
 
 
 
 
 
 
 
 
 
 
 
 
 
 
 
 
 
 
 
 
 
 
 
 
 
 
 
 
 
 
 
 
 
 
 
 
 
 
 
 
 
----------------------------------------------------------
-------------------------  *X*  --------------------------
----------------------------------------------------------
 
 
 
 
 
 
 
 
 
 
 
 
 
 
 
 
 
 
 
 
 
 
 
 
 
 
 
 
 
 
 
 
 
 
 
 
 
 
 
 
 
 
 
 
 
 
 
 
 
 
 
 
 
 
 
 
 
 
 
 
 
 
 
 
 
 
 
 
 
 
 
 
 
 
 
 
 
 
 
 
 
 
 
 
 
 
 
 
 
 
 
 
 
 
 
 
 
----------------------------------------------------------
-------------------------  *Y*  --------------------------
----------------------------------------------------------
 
 
 
 
 
 
 
 
 
 
 
 
 
 
 
 
 
 
 
 
 
 
 
 
 
 
 
 
 
 
 
 
 
 
 
 
 
 
 
 
 
 
 
 
 
 
 
 
 
 
 
 
 
 
 
 
 
 
 
 
 
 
 
 
 
 
 
 
 
 
 
 
 
 
 
 
 
 
 
 
 
 
 
 
 
 
 
 
 
 
 
 
 
 
 
 
 
----------------------------------------------------------
-------------------------  *Z*  --------------------------
----------------------------------------------------------
 
   Available commands:
 
      ABORT     ABRR      ADD
      BOTTOM    BOX
      CALL      CENTER    CHANGE    CA        COPY      CUT
      DELETE    DHOLD     DI        DUP       DTL
      EDIT      EXECUTE   EXTEND    EXIT      EXRR      EXEM
      FIND      FORMAT
      GO
      HELP      HOLD
      INDENT    INPUT     INSERT
      LAST      LOCATE    LI        LC        LIST
                LA        LOBL     LABL
      MOVE
      NEXT
      PRINT     PA        PN        PP        PPAR      PTL
      SET       SHIFT     SHOW      SPLIT
      TOP       TRUNCATE  TRANSLATE
      UNDO      UP
 
   Commands which are not yet operational (February 1, 1986):
 
      FORMAT
      PPAR
      SHIFT
      SPLIT
      TRANSLATE
 
 
 
----------------------------------------------------------
